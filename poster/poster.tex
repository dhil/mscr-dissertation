\documentclass[noback,landscape]{infposter}
\usepackage[utf8]{inputenc}           % Enables UTF-8 encoding
\usepackage{lipsum}
% Source code listings
\usepackage{listings}

\lstset{
 backgroundcolor=\color{white},   % choose the background color; you must add \usepackage{color} or \usepackage{xcolor}
 basicstyle=\ttfamily\footnotesize,        % the size of the fonts that are used for the code
 keywordstyle=\bfseries,
 commentstyle=\itshape,
 % SL: lstinline doesn't work properly if breakatwhitespace is not set to true
 breakatwhitespace=true,         % sets if automatic breaks should only happen at whitespace
 breaklines=true,                 % sets automatic line breaking
 captionpos=b,                    % sets the caption-position to bottom
 escapeinside={\#*}{*\#},          % if you want to add LaTeX within your code
 extendedchars=true,              % lets you use non-ASCII characters; for 8-bits encodings only, does not work with UTF-8
 frame=none,	                   % adds a frame around the code
 keepspaces=true,                 % keeps spaces in text, useful for keeping indentation of code (possibly needs columns=flexible)
 numbers=none,                    % where to put the line-numbers; possible values are (none, left, right)
 rulecolor=\color{black},         % if not set, the frame-color may be changed on line-breaks within not-black text (e.g. comments (green here))
 showspaces=false,                % show spaces everywhere adding particular underscores; it overrides 'showstringspaces'
 showstringspaces=false,          % underline spaces within strings only
 showtabs=false,                  % show tabs within strings adding particular underscores
 tabsize=1,	                   % sets default tabsize to 2 spaces
 title=\lstname,                   % show the filename of files included with \lstinputlisting; also try caption instead of title
 caption={},
 belowcaptionskip=-1\baselineskip,
 xleftmargin=0.1\parindent,
 columns=fullflexible
}

\definecolor{darkgreen}{rgb}{0.000000,0.392157,0.000000}
\definecolor{violetred}{rgb}{0.815686,0.125490,0.564706}

% Define Links as a lst-language
\lstdefinelanguage{Links}{% 
  morekeywords={typename, fun, op, var, if, this, true, false, else, case, switch, handle, handler, shallowhandler, open, do, sig},%
  sensitive=t, % 
  keywordstyle=\color{red},
  emph={Comp,Player,Bool,Int,GTree,Cheat,Zero,Choose,Rand,Move,Winner,Take,Return,Get,Put,GameState,Alice,Bob,Fork,Yield},
  emphstyle={\color{blue}},
  comment=[l]{\#},% 
  escapeinside={(*}{*)},%
  morestring=[d]{"}%
}

\newcommand{\textapprox}{{\fontfamily{ptm}\selectfont\texttildelow}}
\newcommand{\wildarrow}{\linksify{\textapprox{}>}}
% Links style
\lstdefinestyle{links}{
  basicstyle=\linespread{1.0}\ttfamily,
  language=Links,
  literate= {~>}{{\wildarrow}}1
}

\lstset{style={links}}

\title{Towards Compilation of Affine Algebraic Effect Handlers}
%\subtitle{From theory to practice}
\author{Daniel Hillerström}
\email{daniel.hillerstrom@ed.ac.uk}
\homepage{http://homepages.inf.ed.ac.uk/s1467124}

\footer{%
  \resizebox{!}{\footerheight}{\includegraphics{logo_icsa.eps}}%
  \hfill %
  \resizebox{!}{\footerheight}{\includegraphics{logo_lfcs.eps}} %
  \hfill %
  \resizebox{!}{\footerheight}{\includegraphics{cdt_pervasive_parallelism.eps}} %
  \hfill %
  \resizebox{!}{\footerheight}{\includegraphics{school_of_informatics.eps}}%
}

\abstract{%
Algebraic effects and effect handlers provide a modular abstraction
for effectful programming.
%
They support user-defined effects, as in Haskell, in conjunction with
direct-style effectful programming, as in ML.
%
They also present a structured interface to programming with delimited
continuations.
}

\begin{document}
\makeposter

\section{Concurrency as an effect}
An algebraic effect is described by a collection of \emph{abstract} operations. For instance, we may describe concurrency by two operations:
\[
  \emph{concur} = \{\texttt{\lstinline$Fork : (() -> ()) \{\}-> ()$}, \texttt{\lstinline$Yield : ()$}\}
\]

\section{An abstract concurrent program}
\begin{lstlisting}
BLAH BLAH
\end{lstlisting}

\section{Handling concurrency}
A handler instantiates \emph{abstract} operations with concrete implementations.
\begin{lstlisting}[numbers=left]
fun pspawn(task) {
  handle(task) {
    case Return(x)     -> 
      dequeue()()    # Task finished, run the next task
    case Fork(child,k) -> 
      enqueue(k);    # Suspend parent task
      pspawn(child)  # Recursively handle the child task
    case Yield(k)      -> 
      enqueue(k);    # Suspend the current task
      dequeue()()    # Run the next task
} }
\end{lstlisting}
\begin{description}
\item[Line 1] gives the type of \lstinline$perfect$: it takes a
  computation that may invoke the \lstinline$Move$ operation and
  returns a value of type \lstinline$a$.

\item[Lines 2 and 3] begin the definition. The function
  \lstinline$perfect$ wraps the actual handler, which is applied to
  the argument $m$ using the \lstinline$handle$ construct, which
  specifies how to interpret abstract operations through a sequence of
  clauses.

\item[Line 4] is a \emph{return clause}. It defines how to handle the
  final return value of the input computation. In this case, this
  value is simply returned as is.

\item[Line 5] is an \emph{operation clause}. It expresses how to
  handle \lstinline$Move$. In general, an operation clause takes the
  form $Op(p_1,\dots,p_n,k) \to M$, where $p_1,\dots,p_n$ are patterns
  that bind the operation parameters and $k$ is a pattern that binds
  the continuation of the computation in $M$. In this case
  \lstinline$p$ and \lstinline$n$ are bound to the active player and
  number of sticks in the heap, respectively. The continuation is
  invoked with the perfect strategy, irrespective of the player.
\end{description}
\vfill

\section{Compiler infrastructure}

\section{Preliminary results}
\lipsum[75]
\section{Future work}

\begin{itemize}
  \item Implement handler-oriented optimisations.
  \item Reconstruct Links' message-passing concurrency model.
\end{itemize}

\section{Acknowledgements}
Thanks to OCaml Labs\dots
This work was supported in part by the EPSRC Centre for Doctoral Training in Pervasive Parallelism, funded by the UK Engineering and Physical Sciences Research Council (grant EP/L01503X/1) and the University of Edinburgh.
\end{document}
