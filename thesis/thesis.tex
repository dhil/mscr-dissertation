%% Master by Research, ICSA - LFCS, print twosided, new chapters on right page
\documentclass[mscres,icsa,lfcs,twoside,openright,logo,rightchapter,normalheadings]{infthesis}

%% Packages
\usepackage[utf8]{inputenc}   % Enable UTF-8 typing
\usepackage[british]{babel}   % British English
\usepackage[breaklinks]{hyperref}         % Interactive PDF
\usepackage{url}
\usepackage{breakurl}
\usepackage{amsmath}          % Mathematics library
\usepackage{amssymb}          % Provides math fonts
\usepackage{amsthm}           % Provides \newtheorem, \theoremstyle, etc.
\usepackage{mathpartir}       % Inference rules
\usepackage{array}
\usepackage{float}            % Float control
\usepackage{caption,subcaption}  % Sub figures support
\usepackage[T1]{fontenc}      % Fixes font issues
\usepackage{lmodern}

% Drawing
\usepackage{tikz}
\usetikzlibrary{trees}
\usetikzlibrary{calc}  
\usetikzlibrary{fit}
\usetikzlibrary{shapes}

% Bibliography
\usepackage{natbib}
% \usepackage[
% backend=bibtex,
% style=numeric,
% sorting=nyt
% ]{biblatex}  
% \addbibresource{references.bib}

% Source code listings
\usepackage{xcolor}
\usepackage{listings}

\lstset{
 backgroundcolor=\color{white},   % choose the background color; you must add \usepackage{color} or \usepackage{xcolor}
 basicstyle=\ttfamily\small,        % the size of the fonts that are used for the code
 commentstyle=\itshape,
 breakatwhitespace=false,         % sets if automatic breaks should only happen at whitespace
 breaklines=true,                 % sets automatic line breaking
 captionpos=b,                    % sets the caption-position to bottom
 deletekeywords={...},            % if you want to delete keywords from the given language
 escapeinside={§*}{*§},          % if you want to add LaTeX within your code
 extendedchars=true,              % lets you use non-ASCII characters; for 8-bits encodings only, does not work with UTF-8
 frame=none,	                   % adds a frame around the code
 keepspaces=true,                 % keeps spaces in text, useful for keeping indentation of code (possibly needs columns=flexible)
 numbers=none,                    % where to put the line-numbers; possible values are (none, left, right)
 rulecolor=\color{black},         % if not set, the frame-color may be changed on line-breaks within not-black text (e.g. comments (green here))
 showspaces=false,                % show spaces everywhere adding particular underscores; it overrides 'showstringspaces'
 showstringspaces=false,          % underline spaces within strings only
 showtabs=false,                  % show tabs within strings adding particular underscores
 tabsize=2,	                   % sets default tabsize to 2 spaces
 title=\lstname,                   % show the filename of files included with \lstinputlisting; also try caption instead of title
  belowcaptionskip=-1\baselineskip,
  xleftmargin=\parindent
}

% Define Links as a lst-language
\lstdefinelanguage{Links}{% 
  morekeywords={typename, fun, op, var, if, this, true, false, else, case, switch, handle, handler, shallowhandler, open, do, sig},%
  sensitive=t, % 
  keywordstyle=\color{red},
  emph={Comp,Player,Bool,Int,GTree,Cheat,Zero,Choose,Rand,Move,Winner,Take,Return,Get,Put,GameState,Alice,Bob},
  emphstyle={\color{blue}},
  comment=[l]{\#},% 
  escapeinside={(*}{*)},%
  morestring=[d]{"}%
 }

% Haskell style
\lstdefinestyle{haskell}{
  language=Haskell,
  basicstyle=\linespread{1.0}\ttfamily\footnotesize,
  literate= {+}{{$+$}}1 {*}{{$*$}}1
            {<=}{{$\leq$}}1 {/=}{{$\neq$}}1 
            {==}{{$\equiv$}}1 {=>}{{$\Rightarrow$}}1
            {->}{{$\to$}}1 {<-}{{$\leftarrow$}}1
            {.}{{$\circ$}}1 {$$}{{\$}}1
}
% Ocaml style
\lstdefinestyle{ocaml}{
  language=Caml,
  literate= {+}{{$+$}}1 {*}{{$*$}}1
            {<=}{{$\leq$}}1 {>=}{{$\geq$}}1 {<>}{{$\neq$}}1 
            {==}{{$\equiv$}}1 {=>}{{$\Rightarrow$}}1
            {->}{{$\to$}}1
}

\newcommand{\textapprox}{{\fontfamily{ptm}\selectfont\texttildelow}}
\newcommand{\wildarrow}{\linksify{\textapprox{}>}}
% Links style
\lstdefinestyle{links}{
  basicstyle=\linespread{1.0}\ttfamily\footnotesize,
  language=Links,
  literate= {~>}{{\wildarrow}}1
}

\lstset{style={links}}



% Theorem environments
\newtheorem{theorem}{Theorem}[section]
\newtheorem{lemma}[theorem]{Lemma}
\newtheorem{proposition}[theorem]{Proposition}
\newtheorem{corollary}[theorem]{Corollary}
\newtheorem{definition}[theorem]{Definition}

% Example environment
\makeatletter  
\def\@endtheorem{\qed\endtrivlist\@endpefalse } % insert `\qed` macro
\makeatother
\theoremstyle{definition}
\newtheorem{example}{Example}[chapter]

%% Utilities
% TODOs
\newcommand{\todo}[1]{{\par\noindent\small\color{red} \framebox{\parbox{\dimexpr\linewidth-2\fboxsep-2\fboxrule}{\textbf{TODO:} #1}}}}

% Convenient macros
\newcommand{\linksify}[1]{\texttt{#1}}
\newcommand{\defas}[0]{\mathrel{\overset{\makebox[0pt]{\mbox{\normalfont\tiny\sffamily def}}}{=}}} % "defined-as-equal"
\newcommand{\rulesep}{\unskip\ \vrule\ } % Inserts a vertical line

% Formalisation
%\newcommand{\Calc}{\ensuremath{\lambda_{\text{effrow}}}\xspace}
\newcommand{\Calc}{\ensuremath{\lambda_{\text{eff}}^\rho}\xspace}

\newcommand{\slab}[1]{\textrm{#1}}
\newcommand{\semlab}[1]{\text{\scshape{S-#1}}}
\newcommand{\tylab}[1]{\text{\scshape{T-#1}}}
\newcommand{\mlab}[1]{\text{\scshape{M-#1}}}
\newcommand{\klab}[1]{#1}
\newcommand{\revto}{\ensuremath{\leftarrow}}

\newcommand{\keyw}[1]{\textbf{#1}}
\newcommand{\Handle}{\keyw{handle}}
\newcommand{\With}{\keyw{with}}
\newcommand{\Let}{\keyw{let}}
\newcommand{\In}{\keyw{in}}
\newcommand{\Do}{\keyw{do}}
\newcommand{\Return}{\keyw{return}}
\newcommand{\Case}{\keyw{case}}
\newcommand{\Absurd}{\keyw{absurd}}
\newcommand{\Record}[1]{\ensuremath{\langle #1 \rangle}}

\newcommand{\Pre}[1]{\mathsf{Pre}(#1)}
\newcommand{\Abs}{\mathsf{Abs}}
\newcommand{\Presence}{\mathsf{Presence}}
\newcommand{\Row}{\mathsf{Row}}
\newcommand{\Type}{\mathsf{Type}}

\newcommand{\Int}{\mathsf{Int}}
\newcommand{\Bool}{\mathsf{Bool}}

\newcommand{\True}{\mathsf{true}}
\newcommand{\False}{\mathsf{false}}

\newcommand{\cek}[1]{\ensuremath{\langle #1 \rangle}}

% typing
\newcommand{\eff}{\mathbin{!}}
\newcommand{\typ}[3]{#1 \vdash #2 \eff #3}

\newcommand{\harrow}[2]{\ensuremath{\mathbin{~^{#1}\!\!\Rightarrow^{#2}}}}
%\newcommand{\Harrow}[4]{#1 \harrow{#2}{#4} #3}
\newcommand{\Harrow}[4]{#1!#2 \Rightarrow #3!#4}

% stacks
\newcommand{\nil}{\ensuremath{[\,]}}
\newcommand{\cons}{\ensuremath{::}}

% These operators are taken from Conor's Agda course:
%   * fish appends a cons list onto a snoc list by reversing the cons list
%     (yielding a snoc list)
%   * chips appends a cons list onto a snoc list by reversing the snoc list
%     (yielding a cons list)
%   * chips is revapp (assuming we represent snoc lists as cons lists)
\newcommand{\fish}{\ensuremath{\mathbin{<\!>\!\!<}}}
\newcommand{\chips}{\ensuremath{\mathbin{<\!>\!\!>}}}

%\newcommand{\revapp}{\ensuremath{\mathbin{<\!\!\!+\!\!+}}}
\newcommand{\revapp}{\chips}

\newcommand{\concat}{\mathbin{+\!\!+}}


%% Effects on the turnstyle
%\newcommand{\typ}[3]{#1 \vdash_{#3} #2}


% environments
\newcommand{\env}{\gamma}

\newcommand{\reducesto}[0]{\ensuremath{\leadsto}}
\newcommand{\stepsto}[0]{\ensuremath{\longrightarrow}}
\newcommand{\Stepsto}{\Longrightarrow}

% array stuff
\newcommand{\ba}{\begin{array}}
\newcommand{\ea}{\end{array}}

\newcommand{\bl}{\ba[t]{@{}l@{}}}
\newcommand{\el}{\ea}

% Syntax environment
\newenvironment{syntax}{\[\ba{@{}l@{\quad}r@{~}c@{~}l@{}}}{\ea\]\ignorespacesafterend}
\newenvironment{reductions}{\[\ba{@{}l@{\qquad}@{}r@{~~}c@{~~}l@{}}}{\ea\]\ignorespacesafterend}

\newenvironment{eqs}{\ba{@{}r@{~}c@{~}l@{}}}{\ea}
\newenvironment{equations}{\[\ba{@{}r@{~}c@{~}l@{}}}{\ea\]\ignorespacesafterend}


% translations
\newcommand{\val}[2]{\llbracket #1 \rrbracket #2}
\newcommand{\inv}[1]{\llparenthesis #1 \rrparenthesis}

% restrict an environment
\newcommand{\res}{\backslash}

%% Information about the title, etc.
\title{Compilation of Affine Algebraic Effect Handlers}
\author{Daniel Hillerström}

%% If the year of submission is not the current year, uncomment this line and 
%% specify it here:
\submityear{2016}

%% Specify the abstract here.
\abstract{%
  \todo{Refine abstract} 
%
  Effectful operations, such as raising exceptions, thread forking, writing and reading from files or updating some mutable state, are pervasive in applications.

% 
  Many programs comprise effectful operations such as thread forking,
  raising exceptions, writing and reading from files or updating some
  mutable state.  have a salient impact on the behaviour of programs.
  During run-time programs may perform several side-effecting
  actions. Effects are a salient part of Programs Algebraic effects
  combined with effect handlers provide a modular abstraction for
  effectful programming.
%
Concurrency and parallelism 
%%
The key insight is to classify handlers according to their linearity.
%%

We present a core calculus \Calc{} with row-polymorphic effects and affine handlers based on a variation of A-normal form used in our implementation. In addition, we give an operational semantics for the calculus, which we use to prove the soundness of the calculus.
}

%% Now we start with the actual document.
\begin{document}
\raggedbottom
%% First, the preliminary pages
\begin{preliminary}

%% This creates the title page
\maketitle

%% Acknowledgements
\begin{acknowledgements}
Acknowledgements will appear here\dots

Thanks Christophe\dots

Thanks Sam\dots

Thanks everyone\dots
\end{acknowledgements}

%% Next we need to have the declaration.
\standarddeclaration

%% Finally, a dedication (this is optional -- uncomment the following line if
%% you want one).
%\dedication{To my mummy.}

\begin{preface}
A preface will possibly appear here\dots
\end{preface}

%% Create the table of contents
\setcounter{secnumdepth}{2} % Numbering on sections and subsections
\setcounter{tocdepth}{3} % Show chapters, sections and subsections in TOC
%\singlespace
\tableofcontents
%\doublespace

%% If you want a list of figures or tables, uncomment the appropriate line(s)
% \listoffigures
% \listoftables
\end{preliminary}

%%%%%%%%%%%%%%%%%%%%%%%%%%%%%%%%%%%
%%          Main content         %%
%%%%%%%%%%%%%%%%%%%%%%%%%%%%%%%%%%%

%%
%% Introduction
%%
\chapter{Introduction}
\section{Problem analysis}
\section{Problem statement}
\section{Aims and objectives}
\section{Thesis outline}

%%
%% Background
%%
\chapter{Background}
\section{The Links programming language}
\subsection{Notable features}
\subsection{Concurrency model}
\subsection{Programming with handlers}

\section{Algebraic effects and handlers}
\subsection{Algebraic effects}
\subsection{Effect handlers}

\section{Related work}
\subsection{Implementations of effect handlers}
\subsection{Pragma directives}


%%
%% Concurrent programming with handlers
%%
\chapter{Concurrent programming with handlers}

%%
%% Compiling handlers
%%
\chapter{Compiling handlers}

%%
%% Formalisation
%%
\chapter{Formalisation}
In this chapter, we present a type and effect system and a small-step
operational semantics for \Calc (pronounced ``lambda-eff-row''), a
Church-style row-polymorphic call-by-value calculus for effect
handlers.
%
We prove that the operational semantics is sound with respect to the
type and effect system.

A key advantage of row polymorphism is that it integrates rather
smoothly with Hindley-Milner type inference. We concern ourselves only
with the explicitly-typed core language, as the treatment of type
inference is quite standard.

The design of \Calc is inspired by the $\lambda$-calculi of
\citet{Kammar2013}, \citet{Pretnar2015}, and \citet{Lindley2012}.
%
As in the work of \citet{Kammar2013}, each handler can have its own
effect signature. As in the work of \citet{Pretnar2015}, the
underlying formalism is fine-grained call-by-value~\cite{LevyPT03},
which names each intermediate computation like in A-normal
form~\cite{Flanagan1993}, but unlike A-normal form is closed under
$\beta$-reduction. As in the work of \citet{Lindley2012}, the effect
system is based on row polymorphism.

\section{Types}
The grammars of types, effects, kinds, and type and kind environments
are given in Figure \ref{fig:types-syntax}.

\begin{figure}
Types
\begin{syntax}
\slab{Ordinary types} &A,B,C  &::= & %\Bool \mid \Int
                                      A \xrightarrow{E} B
                               \mid  \forall \alpha^K .\, A \\
                             &&\mid& \Record{R} \mid [R]
                               \mid  \Harrow{A}{E}{B}{E'} \mid \alpha \\
\slab{Effect types}   &E    &::= & \{R\}\\
\slab{Row types}      &R    &::= & \ell : P;R \mid \rho \mid \cdot \\
\slab{Presence types} &P    &::= & \Pre{A} \mid \Abs \mid \theta\\
%\slab{Labels}         &\ell &    &                \\
\slab{Kinds}          &K    &::= & \Type \mid \Row_\mathcal{L} \mid \Presence\\
\slab{Label sets}     &\mathcal{L} &::=& \emptyset \mid \{\ell\} \uplus \mathcal{L}\\
%\slab{Type variables} &\alpha, \rho, \theta& \\
\slab{Type environments} &\Gamma &::=& \cdot \mid \Gamma, x:A \\
\slab{Kind environments} &\Delta &::=& \cdot \mid \Delta, \alpha:K \\
\end{syntax}
\caption{Types, effects, kinds, and environments}
\label{fig:types-syntax}
\end{figure}

\paragraph{Ordinary Types}
%The base types are integers and booleans.
The function type $A \xrightarrow{E} B$ takes an argument of type $A$,
returns a value of type $B$, and has effect signature $E$. The effect
signature enumerates the operations that the function may perform
during evaluation. The polymorphic type $\forall \alpha^K .\, A$ is
parameterised by a type variable $\alpha$ of kind $K$. The record type
$\Record{R}$ represents records with fields given by labels of row
$R$. Dually, the variant type $[R]$ represents a sum of fields tagged
by the labels of row $R$. The handler type $\Harrow{A}{E}{B}{E'}$
transforms a computation with return type $A$, that has effect
signature $E$, into another computation with return type $B$ and
effect signature $E'$.

\paragraph{Row Types}
Effect signatures, records and variants are defined in terms of rows.
A row type embodies a collection of distinct labels, each of which is
annotated with a presence type. A presence type indicates whether a
label is \emph{present} with some type $A$ ($\Pre{A}$), \emph{absent}
($\Abs$) or \emph{polymorphic} in its presence ($\theta$).

Row types are either \emph{closed} or \emph{open}. A closed row type
ends in $\cdot$, whilst an open row type ends with a \emph{row
  variable} $\rho$. Furthermore, a closed row term can have only the
labels explicitly mentioned in its type. Conversely, the row variable
in an open row can be instantiated with additional labels. We identify
rows up to reordering of labels, for instance, we consider the
following two rows equivalent:
\[ \ell_1 : P_1; \cdots; \ell_n : P_n \equiv \ell_n : P_n; \cdots ; \ell_1 : P_1. \]
The unit and empty type are definable in terms of row types. We define
the unit type as the empty, closed record, that is,
$\Record{\cdot}$. Similarly, we define the empty type as the empty,
closed variant $[\cdot]$. Usually, we usually omit the $\cdot$ for
closed rows.

\paragraph{Kinds}
We have three kinds: $Type$, $\emph{Row}_\mathcal{L}$ and $Presence$
which classify ordinary types, row types and presence types,
respectively. Row kinds are annotated with a set of labels
$\mathcal{L}$. The kind of a complete row is
$Row_{\mathcal{\emptyset}}$. More generally, the kind
$Row_{\mathcal{L}}$ denotes a partial row which cannot mention the
labels in $\mathcal{L}$.
%

\paragraph{Type Variables}
We let $\alpha$, $\rho$ and $\theta$ range over type variables. By
convention we use $\alpha$ for ordinary type variables or for type
variables of unspecified kind, $\rho$ for type variables of row kind,
and $\theta$ for type variables of presence kind.

\paragraph{Type and Kind Environments}
Type environments map term variables to their types and kind
environments map type variables to their kinds.

\section{Terms}
\begin{figure}
\begin{syntax}
                             % SL: no need for constants in the core calculus
%\slab{Constants}     &c    &::= & n \mid \True \mid \False \\
\slab{Values}        &V,W  &::= & x
                             %\mid c
                             \mid \lambda x^A .\, M \mid \Lambda \alpha^K .\, M  \\
                     &     &\mid& \Record{} \mid \Record{\ell = V;W} \mid (\ell\, V)^R \\
                     &     &    &\\
\slab{Computations}  &M,N  &::= & V\,W \mid V\,A\\
                     &     &\mid& \Let\; \Record{\ell=x;y} \revto V \; \In \; N\\
                     &     &\mid& \Case\; V \{\ell\; x \mapsto M; y \mapsto N\} \mid \Absurd^A V\\
                     &     &\mid& \Return\; V \\
                     &     &\mid& \Let \; x \revto M \; \In \; N\\
                     &     &\mid& (\Do \; \ell \; V)^E \\
                     &     &\mid& \Handle \; M \; \With \; H\\
                     &     &    &\\
\slab{Handlers}      &H    &::= & \{ \Return \; x \mapsto M \} \\
                     &     &\mid& \{ \ell \; x \; k \mapsto M \} \uplus H \\
%% \slab{Integers}      &n    &      \\
%% \slab{Variables}     &x,y,z,k &    & \\
\end{syntax}

\caption{Term Syntax}
\label{fig:term-syntax}
\end{figure}
The terms are given in Figure \ref{fig:term-syntax}. We let $x,y,z,k$
range over term variables. By convention, we use $k$ to denote names
of continuations. %We let $n$ range over integer constants.

The syntax partitions terms into values, computations and
handlers. 
%
Value terms comprise variables ($x$), %constants ($c$),
lambda abstraction ($\lambda x^A . \, M$), type abstraction ($\Lambda
\alpha^K . \, M$), and the introduction forms for records and
variants. Records are introduced using the empty record $\Record{}$
and record extension $\Record{\ell = V; W}$, whilst variants are
introduced using injection $(\ell\, V)^R$ which injects a field with
label $\ell$ and value $V$ into a row whose type is $R$. We include
the row type annotation in order to support bottom-up type
reconstruction.

All elimination forms are computation terms. Abstraction and type
abstraction are eliminated using application ($V\,W$) and type
application ($V\,A$) respectively.
%
The record eliminator $(\Let \; \Record{\ell=x;y} \revto V \; \In \;
N)$ splits a record $V$ into $x$, the value associated with $\ell$,
and $y$, the rest of the record. Non-empty variants are eliminated
using the case construct ($\Case\; V\; \{\ell\; x \mapsto M; y \mapsto
N\}$), which evaluates the computation $M$ if the tag of $V$ matches
$\ell$, otherwise it falls through to $y$ and evaluates $N$.  The
elimination form for empty variants is ($\Absurd^A V$). A trivial
computation $(\Return\;V)$ returns value $V$. The expression $(\Let \;
x \revto M \; \In \; N)$ evaluates $M$ and binds the result value to
$x$ in $N$.

The construct $(\Do \; \ell \; V)^E$ invokes an operation $\ell$ with
value argument $V$. The handle construct $(\Handle \; M \; \With \;
H)$ runs a computation $M$ with handler definition $H$. A handler
definition $H$ consists of a return clause $\Return \; x \mapsto M$
and a possibly empty set of operation clauses $\{\ell_i \; x_i \; k_i
\mapsto M_i\}_i$. The return clause defines how to handle the final
return value of the handled computation, which is bound to $x$ in $M$.
%
The $i$-th operation clause binds the operation parameter to $x_i$ and
a the continuation $k_i$ in $M_i$.

We write $H(\Return)$ for the return clause of $H$ and $H(\ell)$ for
the set of either zero or one operation clauses in $H$ that handle the
operation $\ell$. We write $dom(H)$ for the set of operations handled
by $H$.
%
We annotate various subterms with their types in order to aid type
reconstruction (injection, operations, empty cases, and handlers); we
sometimes omit these annotations.

%% SL: possibly a bit draconian, but I think we can do without the
%% detail here.

%% The terms of \Calc are written in a style reminiscent of
%% \emph{A-normal form} (ANF). ANF is a direct style intermediate
%% language in which every intermediate computation is let-bound
%% \cite{Flanagan1993}. As a consequence application consist of a value
%% applied to another value. For example, consider the Links expression
%% \lstinline$f(g(42))$, we must let bind the argument of \lstinline$f$
%% in order to bring the expression into ANF; we do not need to let bind
%% the argument of \lstinline$g$, since it is already a value, thus we
%% obtain the following ANF expression:
%% \[
%% \Let\; x \revto g\,42\; \In \; f\, x.
%% \]
%% However, ANF is not closed under $\beta$-reduction
%% \cite{Sabry1996,Kennedy2007}, since function application can lead to
%% nested \Let{}s. Consider the following example, which is adapted from
%% \citet{Sabry1996}:
%% \[ 
%%   \Let\; x \revto (\lambda y . \, \Let\; z \revto f \, a \;\In \; M) \, b \; \In\; N
%% \]
%% which $\beta$-reduces to 
%% \[
%%   \Let\; x \revto (\Let\; z \revto f \, a \;\In \; M) \; \In\; N
%% \]
%% which is not in ANF, because $x$ is bound to a computation rather than
%% a value. ANF can be recovered by adding a normalisation step
%% \cite{Sabry1996}.  We simplify our presentation by permitting nested
%% let-bindings and handlers to be applied directly to computations. The
%% resulting calculus is closer to fine-grained
%% call-by-value~\cite{LevyPT03}.

%% ,
%% therefore we say that \Calc-terms are in \emph{relaxed A-normal
%%   form} (RANF).

% As an example consider the translation of the Links expression \linksify{f(g(42),\{ var y = 1; h(y) \})} into its ANF. For clarity, we highlight the reducible term with a box:
% \begin{align*}
%             &  \fbox{\linksify{f(g(42),\{ var y = true; h(y) \})}}\\
% \Rightarrow &\quad\Let \; x_1 \revto g \; 42 \; \In\\
%             &\quad\quad\fbox{\linksify{f($x_1$,\{ var y = true; h(y) \})}}\\
% \Rightarrow &\quad\Let \; x_1 \revto g \; 42 \; \In\\
%             &\quad\quad\Let \; y \revto \Return \; \True \; \In\\
%             &\quad\quad\quad\fbox{\linksify{f($x_1$,h($y$))}}\\
% \Rightarrow &\quad\Let \; x_1 \revto g \; 42 \; \In\\
%             &\quad\quad\Let \; y \revto \Return \; \True \; \In\\
%             &\quad\quad\quad\Let \; x_2 \revto h \; y \; \In\\
%             &\quad\quad\quad\quad(f \; x_1)\; x_2
% \end{align*}
% Accordingly, the function \linksify{f} is applied only to value terms. Every subexpression gets decomposed into a single value gradually. As a consequence every subexpression has been explicitly named, thus in this respect, ANF is similar to the predominant imperative Static Single Assignment form.

\section{Static Semantics}
\label{sec:typing}
The kinding rules are given in Figure \ref{fig:kinding} and the typing
rules are given in Figure \ref{fig:typing}.

The kinding judgement $\Delta \vdash \alpha : K$ asserts that the type
variable $\alpha$ has kind $K$ in kind environment $\Delta$. The
typing judgement $\typ{\Delta;\Gamma}{M : A}{E}$ states that the term
$M$ has type $A$ and effects $E$ under kind environment $\Delta$ and
type environment $\Gamma$. In typing judgements, we implicitly assume
that $\Gamma$, $E$ and $A$ are well-kinded with respect to
$\Delta$. We define the functions $FTV(\Gamma)$ and $FTV(E)$ to be the
set of free type variables in $\Gamma$ and $E$, respectively. We write
$FTV(\Gamma,E)$ as shorthand for $FTV(\Gamma) \cup FTV(E)$.

The kind and typing rules are mostly straightforward. The interesting
typing rules are \tylab{Handle} and the two handler rules. The
\tylab{Handle} rule states that $\Handle\; M\; \With\; H$ produces a
computation of type $B$ given that the computation $M$ is typeable
under effect context $E$, and that $H$ is a handler which transforms a
computation of type $A$ with effect signature $E$ into another
computation of type $B$ with effect signature $E'$.

The \tylab{Handler} rule is crucial. The input effect $E$ and the
output effect $E'$ must share the same suffix $E_f$. This means that
$E''$ must explicitly mention each of the operations $\ell_i$, whether
that be to say that an $\ell_i$ is present with a given type
signature, absent, or polymorphic in its presence. The $f$ in $E_f$
stands for ``forward'', as in the effects that are forwarded by the
handler. The effect $E_f$ may include a row-variable, in which case an
arbitrary number of effects may be forwarded by the handler.
%
The typing of the return clause is straightforward. In the typing of
each operation clause, the continuation is annotated with the output
effect $E'$. Thus, we are here defining \emph{deep}
handlers~\cite{Kammar2013} in which the handler is implicitly wrapped
around the continuation, such that any subsequent operations are
handled uniformly by the same handler.
%
The Links implementation also supports \emph{shallow}
handlers~\cite{Kammar2013}, in which the continuation is instead
annotated with the input effect and one has to explicitly reinvoke the
handler after applying the continuation inside an operation clause.

%% \todo{Insert a forward reference to discussion of shallow handlers
%%   elsewhere.}

%% The \tylab{Deep} and \tylab{Shallow} rules for handlers describe how
%% to type deep and shallow handlers, respectively. The two rules are
%% similar, but differ crucially in their typing of the continuation
%% parameter $k$. The different typing is due to their different
%% behaviour. A deep handler wraps itself around continuations such that
%% any subsequent operations are handled uniformly. Therefore the return
%% type of a continuation type must match the input type of the
%% \Return{} case, and in particular, its effect signature must be the
%% same as the handler's output effect signature $E'$. Conversely, in a
%% shallow handler the continuation is annotated with the input effect
%% signature $E$ rather than the output effect signature. Moreover, the
%% return type $A$ is the same as return type of the handled
%% computation. Shallow handlers do not wrap themselves around the
%% continuation, therefore the continuation must be applied explicitly to
%% handler which has to handle the next operation
%% invocation. Consequently, operations may be handled non-uniformly.

%% For both handler types the input effect signature is a disjoint union
%% of operation signatures, that the handler directly interprets, and
%% some set $E_f$. The metavariable $E_f$ can be instantiated with a row
%% variable to include operations that are not explicitly mentioned by
%% the handler.

\begin{figure}
\begin{mathpar}
% alpha : K
  \inferrule*[Lab=\klab{TyVar}]
    { }
    {\Delta, \alpha : K \vdash \alpha : K}

% forall alpha : K . A : Type
  \inferrule*[Lab=\klab{Forall}]
    { \Delta, \alpha : K \vdash A : \Type }
    {\Delta \vdash (\forall \alpha^K . \, A) : \Type}
%%
%% % Int : Type
%%   \inferrule*[Lab=\klab{Int}]
%%     { }
%%     {\Delta \vdash \Int : \Type}
%%
%% % Bool : Type
%%   \inferrule*[Lab=\klab{Bool}]
%%     { }
%%     {\Delta \vdash \Bool : \Type}

% A -E-> B, A : Type, E : Row, B : Type
  \inferrule*[Lab=\klab{Fun}]
    { \Delta \vdash A : \Type \\
      \Delta \vdash E : \Row_\emptyset  \\
      \Delta \vdash B : \Type
    }
    {\Delta \vdash A \xrightarrow{E} B : \Type}

% Record
  \inferrule*[Lab=\klab{Record}]
    { \Delta \vdash R : \Row_\emptyset}
    {\Delta \vdash \Record{R} : \Type}

% Variant
  \inferrule*[Lab=\klab{Variant}]
    { \Delta \vdash R : \Row_\emptyset}
    {\Delta \vdash [R] : \Type}

% Present
  \inferrule*[Lab=\klab{Present}]
    {\Delta \vdash A : \Type}
    {\Delta \vdash \Pre{A} : \Presence}

% Absent
  \inferrule*[Lab=\klab{Absent}]
    { }
    {\Delta \vdash \Abs : \Presence}

% Empty row
  \inferrule*[Lab=\klab{EmptyRow}]
    { }
    {\Delta \vdash \cdot : \Row_\mathcal{L}}

% Extend row
  \inferrule*[Lab=\klab{ExtendRow}]
    { \Delta, P : \Presence \\
      \Delta, R : \Row_{\mathcal{L} \uplus \{\ell\}}
    }
    {\Delta \vdash \Record{\ell : P;R} : \Row_\mathcal{L}}
\end{mathpar}

\caption{Kinding Rules}
\label{fig:kinding}
\end{figure}

\begin{figure*}
Values
\begin{mathpar}
% Variable
  \inferrule*[Lab=\tylab{Var}]
    {x : A \in \Gamma}
    {\typ{\Delta;\Gamma}{x : A}{E}}
%%
%% % true : Bool
%%   \inferrule*[Lab=\tylab{True}]
%%     { }
%%     {\typ{\Delta;\Gamma}{\True : \Bool}{E}}
%%
%% % false : Bool
%%   \inferrule*[Lab=\tylab{False}]
%%     { }
%%     {\typ{\Delta;\Gamma}{\False : \Bool}{E}}
%%
%% % n : Int
%%   \inferrule*[Lab=\tylab{Int}]
%%     { n \in \mathbb{N} }
%%     {\typ{\Delta;\Gamma}{n : \Int}{E}}

% Abstraction
  \inferrule*[Lab=\tylab{Lam}]
    {\typ{\Delta;\Gamma, x : A}{M : B}{E}}
    {\typ{\Delta;\Gamma}{\lambda x^A .\, M : A \xrightarrow{E} B}{E}}

% Polymorphic abstraction
  \inferrule*[Lab=\tylab{PolyLam}]
    {\typ{\Delta,\alpha : K;\Gamma}{M : A}{E}  \\
     \alpha \notin FTV(\Gamma, E)
    }
    {\typ{\Delta;\Gamma}{\Lambda \alpha^K .\, M : \forall \alpha^K . \,A}{E}}
\\
% unit : ()
  \inferrule*[Lab=\tylab{Unit}]
    { }
    {\typ{\Delta;\Gamma}{\Record{} : \Record{}}{E}}

% Extension
  \inferrule*[Lab=\tylab{Extend}]
    { \typ{\Delta;\Gamma}{V : A}{E} \\ 
      \typ{\Delta;\Gamma}{W : \Record{\ell:\Abs;R}}{E}
    }
    {\typ{\Delta;\Gamma}{\Record{\ell =V;W} : \Record{\ell:\Pre{A};R}}{E}}

% Inject
  \inferrule*[Lab=\tylab{Inject}]
    {\typ{\Delta;\Gamma}{V : A}{E}}
    {\typ{\Delta;\Gamma}{(\ell\,V)^R : [\ell : \Pre{A}; R]}{E}}
\end{mathpar}
Computations
\begin{mathpar}
% Application
  \inferrule*[Lab=\tylab{App}]
    {\typ{\Delta;\Gamma}{V : A \xrightarrow{E}{E}} B \\
     \typ{\Delta;\Gamma}{W : B}{E}
    }
    {\typ{\Delta;\Gamma}{V\,W : B}{E}}

% Polymorphic application
  \inferrule*[Lab=\tylab{PolyApp}]
    {\typ{\Delta;\Gamma}{V : \forall \alpha^K . \, A}{E} \\
     \typ{\Delta;\Gamma}{\Delta \vdash B : K}{E} \\
     \alpha \notin FTV(E)
    }
    {\typ{\Delta;\Gamma}{V\,B : A[B/\alpha]}{E}}

% Split
  \inferrule*[Lab=\tylab{Split}]
    {\typ{\Delta;\Gamma}{V : \Record{\ell : \Pre{A};R}}{E} \\\\
     \typ{\Delta;\Gamma, x : A, y : \Record{\ell : \Abs; R}}{N : B}{E}
    }
    {\typ{\Delta;\Gamma}{\Let \; \Record{\ell =x;y} \revto V\; \In \; N : B}{E}} 

% Case
  \inferrule*[Lab=\tylab{Case}]
    { \typ{\Delta;\Gamma}{V : [\ell : \Pre{A};R]}{E}  \\\\
      \typ{\Delta;\Gamma,x:A}{M : B}{E} \\\\
      \typ{\Delta;\Gamma,y:[\ell : \Abs;R]}{N : B}{E}
    }
    {\typ{\Delta;\Gamma}{\Case \; V \{\ell\; x \mapsto M;y \mapsto N \} : B}{E}}

% Absurd
  \inferrule*[Lab=\tylab{Absurd}]
    {\typ{\Delta;\Gamma}{V : []}{E}}
    {\typ{\Delta;\Gamma}{\Absurd^A \; V : A}{E}}

% Return
  \inferrule*[Lab=\tylab{Return}]
    {\typ{\Delta;\Gamma}{V : A}{E}}
    {\typ{\Delta;\Gamma}{\Return \; V : A}{E}} 

% Let 
  \inferrule*[Lab=\tylab{Let}]
    {\typ{\Delta;\Gamma}{M : A}{E} \\
     \typ{\Delta;\Gamma, x : A}{N : B}{E}
    }
    {\typ{\Delta;\Gamma}{\Let \; x \revto M\; \In \; N : B}{E}} 
\\
% Do 
  \inferrule*[Lab=\tylab{Do}]
    {\typ{\Delta;\Gamma}{V : A}{E} \\
     E = \{\ell : A \to B; R\}
    }
    {\typ{\Delta;\Gamma}{(\Do \; \ell \; V)^E : B}{E}}

% Open Handle
  \inferrule*[Lab=\tylab{Handle}]
    {\typ{\Delta;\Gamma}{M : A}{E} \\ 
     \Delta;\Gamma \vdash H : \Harrow{A}{E}{B}{E'}}
    {\typ{\Delta;\Gamma}{\Handle \; M \; \With \; H : B}{E'}}
\end{mathpar}

Handlers
\begin{mathpar}
% Closed handler
% \mprset{flushleft}
% \inferrule*[Lab=\tylab{Deep-Closed}]
%     { E = \{\ell_i : A_i \to B_i\}_i \\\\
%       H = \{\Return \; x \mapsto M\} \uplus \{\ell_i \; y_i \; k_i \mapsto N_i\}_i \\\\
%       \left[\Delta;\Gamma,y_i : A_i, k_i : B_i \xrightarrow{E'} C \vdash_{E'} N_i : C\right]_i \\\\
%       \Delta;\Gamma, x : A \vdash_{E'} M : C
%     }
%     {\Delta;\Gamma \vdash_{E'} H : (\Record{} \xrightarrow{E} A) \xrightarrow{E'} C}   

% Open handler
%\mprset{flushleft}
  \inferrule*[Lab=\tylab{Handler}]
    {
      E = \{\ell_i : A_i \to B_i\}_i \uplus E_f \\
      E' = E'' \uplus E_f \\
      H = \{\Return \; x \mapsto M\} \uplus \{\ell_i \; y \; k \mapsto N_i\}_i \\\\
      [\typ{\Delta;\Gamma,y : A_i, k : B_i \xrightarrow{E'} C}{N_i : C}{E'}]_i \\
       \typ{\Delta;\Gamma, x : A}{M : C}{E'} \\
    }
    {{\Delta;\Gamma} \vdash {H : \Harrow{A}{E}{C}{E'}}}

% Shallow closed handler
% \inferrule*[Lab=\tylab{Shallow-Closed}]
%     { E = \{\ell_i : A_i \to B_i\}_i \\\\
%       H = \{\Return \; x \mapsto M\} \uplus \{\ell_i \; y_i \; k_i \mapsto N_i\}_i \\\\
%       \left[\Delta;\Gamma,y_i : A_i, k_i : B_i \xrightarrow{E} A \vdash_{E'} N_i : C\right]_i \\\\
%       \Delta;\Gamma, x : A \vdash_{E'} M : C
%     }
%     {\Delta;\Gamma \vdash_{E'} H : (\Record{} \xrightarrow{E} A) \xrightarrow{E'} C}   

%% SL: we should mention this rule later

%% % Shallow open handler
%%   \inferrule*[Lab=\tylab{Shallow}]
%%     { E = \{\ell_i : A_i \to B_i\}_i \uplus E_f \\\\
%%       E' = E'' \uplus E_f \\\\
%%       H = \{\Return \; x \mapsto M\} \uplus \{\ell_i \; y \; k \mapsto N_i\}_i \\\\
%%       \left[\Delta;\Gamma,y : A_i, k : B_i \xrightarrow{E} A \vdash_{E'} N_i : C\right]_i \\\\
%%       \Delta;\Gamma, x : A \vdash_{E'} M : C
%%     }
%%     {\Delta;\Gamma \vdash_{E'} H : A \harrow{E}{E'} C}
\end{mathpar}

\caption{Typing Rules}
\label{fig:typing}
\end{figure*}

\section{Operational Semantics}
\label{sec:small-step}

\begin{figure*}

\begin{reductions}
\semlab{App}   & (\lambda x^A . \, M) V &\reducesto& M[V/x] \\
\semlab{TyApp} & (\Lambda \alpha^K . \, M) A &\reducesto& M[A/\alpha] \\
\semlab{Split} & \Let \; \Record{\ell = x;y} \revto \Record{\ell = V;W} \; \In \; N &\reducesto& N[V/x,W/y] \\
\semlab{Case$_1$} &
  \Case \; (\ell\, V)^R \{ \ell \; x \mapsto M; y \mapsto N\} &\reducesto& M[V/x] \\
\semlab{Case$_2$} &
  \Case \; (\ell\, V)^R \{ \ell' \; x \mapsto M; y \mapsto N\} &\reducesto& N[(\ell\, V)^R/y], \hfill\quad \text{if } \ell \neq \ell' \\
\semlab{Let} &
  \Let \; x \revto \Return \; V \; \In \; N &\reducesto& N[V/x] \\
\semlab{Handle-Ret} &
  \Handle \; (\Return \; V) \; \With \; H &\reducesto& M[V/x], \hfill\quad \text{where } \{ \Return \; x \mapsto M \} \in H \\
\semlab{Handle-Op} &
  \Handle \; \mathcal{E}[\Do \; \ell \; V] \; \With \; H
    &\reducesto& M[V/x, \lambda y . \, \Handle \; \mathcal{E}[\Return \; y] \; \With \; H/k],\qquad \\
  \multicolumn{4}{@{}r@{}}
      {\text{where } \ell \notin BL(\mathcal{E}) \text{ and } \{ \ell \; x \; k \mapsto M \} \in H} \\
\end{reductions}
\begin{syntax}
\slab{Evaluation contexts} &  \mathcal{E} &::=& [\,] \mid \Let \; x \revto \mathcal{E} \; \In \; N \mid \Handle \; \mathcal{E} \; \With \; H
\end{syntax}
\[
% Evaluation context lift
\inferrule*[Lab=\semlab{Lift}]
  { M \reducesto N }
  { \mathcal{E}[M] \reducesto \mathcal{E}[N]}
\]

\caption{Small-step Operational Semantics}
\label{fig:small-step}
\end{figure*}
We give a small-step operational semantics for \Calc. Figure
\ref{fig:small-step} displays the operational rules. The reduction
relation $\reducesto$ is defined on computation terms. The statement $M
\reducesto M'$ reads: term $M$ reduces to term $M'$ in a single
step. Most of the rules are standard. We use
%% \emph{delimited
%%   computation contexts} and
\emph{evaluation contexts} to simplify the evaluation rules, by
allowing us to focus on an active expression. The interesting rules
are the handler rules.

We write $BL(\mathcal{E})$ for the set of operation labels bound by
$\mathcal{E}$.
\begin{equations}
BL([~])                            &=& \emptyset \\
BL(\Let\;x=\mathcal{E}\;\In\;N)    &=& BL(\mathcal{E}) \\
BL(\Handle\;\mathcal{E}\;\With\;H) &=& BL(\mathcal{E}) \cup dom(H) \\
\end{equations}

The rule \semlab{Handle-Ret} invokes the return clause of a
handler. The rule \semlab{Handle-op} handles an operation by invoking
the appropriate operation clause. The constraint $\ell \notin
BL(\mathcal{E})$ ensures that no inner handler inside the evaluation
context is able to handle the operation: thus a handler is able to
reach past any other inner handlers that do not handle $\ell$. In our
abstract machine semantics we realise this behaviour using explicit
forwarding operations, but more efficient implementations are
perfectly feasible.


We write $R^+$ for the transitive closure of relation $R$.
%
Subject reduction and type soundness for $\Calc$ are standard.

\begin{theorem}[Subject Reduction]
If $\typ{\Delta;\Gamma}{M : A}{E}$ and $M \reducesto M'$, then
$\typ{\Delta;\Gamma}{M' : A}{E}$.
\end{theorem}

There are two ways in which a computation can terminate. It can either
successfully return a value, or it can get stuck on an unhandled
operation.
\begin{definition}
We say that computation term $N$ is normal with respect to effect $E$,
if $N$ is either of the form $\Return\;V$, or
$\mathcal{E}[\Do\;\ell\;W]$, where $\ell \in E$ and $\ell \notin
BL(\mathcal{E})$.
\end{definition}
If $N$ is normal with respect to the empty effect $\{\cdot\}$, then
$N$ has the form $\Return\;V$.

\begin{theorem}[Type Soundness]
If $\typ{}{M : A}{E}$, then there exists $\typ{}{N : A}{E}$, such that
$M \reducesto^+ N \not\reducesto$, and $N$ is normal with respect to
effect $E$.
\end{theorem}

%%
%% Conclusions and future work
%%
\chapter{Conclusions and future work}
\section{Future work}

%%%%%%%%
%% Include your chapter files here. See the sample chapter file for the basic
%% format.

%\section{Introduction}
Points:
\begin{itemize}
  \item Multicore OCaml has handlers but not an effect system
  \item Monads and monad transformers are powerful, but not really flexible, e.g. monad transformer gymnastics.
\end{itemize}

%% If you want the bibliography single-spaced (which is allowed), uncomment
%% the next line.
\nocite{*}
\singlespace
%\printbibliography[heading=bibintoc]
\bibliographystyle{abbrvnat}
\bibliography{references}

%% ... that's all, folks!
\end{document}
