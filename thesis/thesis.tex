%% Master by Research, CDT PPAR, print twosided, new chapters on right page
\documentclass[12pt,mscres,cdtppar,twoside,openright,logo,rightchapter,normalheadings]{infthesis}
\shieldtype{0}

%% Packages
\usepackage[utf8]{inputenc}   % Enable UTF-8 typing
\usepackage[british]{babel}   % British English
\usepackage[breaklinks]{hyperref}         % Interactive PDF
\usepackage{url}
\usepackage{breakurl}
\usepackage{amsmath}          % Mathematics library
\usepackage{amssymb}          % Provides math fonts
\usepackage{amsthm}           % Provides \newtheorem, \theoremstyle, etc.
\usepackage{mathtools}
\usepackage{mathpartir}       % Inference rules
\usepackage{array}
\usepackage{float}            % Float control
\usepackage{caption,subcaption}  % Sub figures support
\usepackage[T1]{fontenc}      % Fixes font issues
\usepackage{lmodern}
\usepackage{enumerate}        % Customise enumerate-environments
\usepackage{xcolor}           % Colours
\usepackage{drawstack}        % Syntactic sugar for using tikz to draw run-time stacks
\usepackage{tikz}
\usetikzlibrary{fit,calc,trees,positioning,arrows,chains,shapes.geometric,%
    decorations.pathreplacing,decorations.pathmorphing,shapes,%
    matrix,shapes.symbols,intersections}

% % Drawing
% \usepackage{tikz}
% \usetikzlibrary{trees}
% \usetikzlibrary{calc}  
% \usetikzlibrary{fit}
% \usetikzlibrary{shapes}

\usepackage{pgf-umlsd}
\usepackage{pgfplots}
\usepgfplotslibrary{fillbetween}

% Bibliography
\usepackage{natbib}
\setcitestyle{authoryear,open={(},close={)}}
% \usepackage[
% backend=bibtex,
% style=numeric,
% sorting=nyt
% ]{biblatex}  
% \addbibresource{references.bib}

% Source code listings
\usepackage{xcolor}
\usepackage{listings}
\usepackage{lstlinebgrd}
\usepackage{expl3,xparse}

\ExplSyntaxOn
\NewDocumentCommand \lstcolorlines { O{orange!15} m }
{
 \clist_if_in:nVT { #2 } { \the\value{lstnumber} }{ \color{#1} }
}
\ExplSyntaxOff

\newcommand{\snippet}[1]{\lstinputlisting{snippets/#1}}

\lstset{
 backgroundcolor=\color{white},   % choose the background color; you must add \usepackage{color} or \usepackage{xcolor}
 basicstyle=\ttfamily\small,        % the size of the fonts that are used for the code
 commentstyle=\itshape,
 breakatwhitespace=false,         % sets if automatic breaks should only happen at whitespace
 breaklines=true,                 % sets automatic line breaking
 captionpos=b,                    % sets the caption-position to bottom
 deletekeywords={...},            % if you want to delete keywords from the given language
 escapeinside={§*}{*§},          % if you want to add LaTeX within your code
 extendedchars=true,              % lets you use non-ASCII characters; for 8-bits encodings only, does not work with UTF-8
 frame=none,	                   % adds a frame around the code
 keepspaces=true,                 % keeps spaces in text, useful for keeping indentation of code (possibly needs columns=flexible)
 numbers=none,                    % where to put the line-numbers; possible values are (none, left, right)
 rulecolor=\color{black},         % if not set, the frame-color may be changed on line-breaks within not-black text (e.g. comments (green here))
 showspaces=false,                % show spaces everywhere adding particular underscores; it overrides 'showstringspaces'
 showstringspaces=false,          % underline spaces within strings only
 showtabs=false,                  % show tabs within strings adding particular underscores
 tabsize=2,	                   % sets default tabsize to 2 spaces
 title=\lstname,                   % show the filename of files included with \lstinputlisting; also try caption instead of title
  belowcaptionskip=-1\baselineskip,
  xleftmargin=\parindent
}

\definecolor{darkgreen}{rgb}{0.000000,0.392157,0.000000}
\definecolor{violetred}{rgb}{0.915686,0.125490,0.364706}

% Define Links as a lst-language
\lstdefinelanguage{Links}{% 
  morekeywords={spawn, receive, typename, fun, op, var, if, this, true, false, else, case, switch, handle, handler, shallowhandler, do, sig, spawnAngel, spawnDemon, spawn},%
  sensitive=t, % 
  keywordstyle=\color{red},
  emph={Comp,Player,Bool,Int,GTree,Cheat,Zero,Choose,Rand,Move,Winner,Take,Return,Get,Put,GameState,Alice,Bob,Fail,Nothing,Just,Maybe,Toss,Heads,Tails,Process,Buyer,Coffee,Pay,Cost,Candidate,Stop,PassingComet,CelebritySighting,Float,String,Pid,EProcess,Row,Spawn,Yield,Recv,Send,FreshName,Queue,Dictionary,Myself},
  emphstyle={\color{blue}},
  comment=[l]{\#},% 
  commentstyle={\itshape\color{darkgreen}},%
  escapeinside={(*}{*)},%
  morestring=[d]{"},%
  stringstyle={\color{violetred}}%
 }

% Haskell style
\lstdefinestyle{haskell}{
  language=Haskell,
  basicstyle=\linespread{1.0}\ttfamily\footnotesize,
  literate= {+}{{$+$}}1 {*}{{$*$}}1
            {<=}{{$\leq$}}1 {/=}{{$\neq$}}1 
            {==}{{$\equiv$}}1 {=>}{{$\Rightarrow$}}1
            {->}{{$\to$}}1 {<-}{{$\leftarrow$}}1
            {.}{{$\circ$}}1 {$$}{{\$}}1
}
% Ocaml style
\lstdefinestyle{ocaml}{
  language=Caml,%
  morekeywords={effect, perform},%
  emph={Obj,Choose},%
%  literate= {+}{{$+$}}1 {*}{{$*$}}1
%            {<=}{{$\leq$}}1 {>=}{{$\geq$}}1 {<>}{{$\neq$}}1 
%            {==}{{$\equiv$}}1 {=>}{{$\Rightarrow$}}1
%            {->}{{$\to$}}1
}

\usepackage{textcomp}
%\newcommand{\textapprox}{{\fontfamily{ptm}\selectfont\texttildelow}}
%\newcommand{\wildarrow}{\linksify{\textapprox{}>}}
\newcommand{\wildarrow}{\fontfamily{ptm}\selectfont\linksify{\textasciitilde{}>}}
% Links style
\lstdefinestyle{links}{
  caption={},
  basicstyle=\linespread{1.0}\ttfamily\footnotesize,
  language=Links,
  literate= {~>}{{\wildarrow}}1
}

\lstset{style={links}}

% Terminal / prompt style
\lstdefinestyle{terminal}{
  caption={},
  basicstyle=\linespread{1.0}\ttfamily\footnotesize,
  keywordstyle={},
  emphstyle={},
  commentstyle={}
}

% Theorem environments
\newtheorem{theorem}{Theorem}[section]
\newtheorem{lemma}[theorem]{Lemma}
\newtheorem{proposition}[theorem]{Proposition}
\newtheorem{corollary}[theorem]{Corollary}
\newtheorem{definition}[theorem]{Definition}

% Example environment
\makeatletter  
\def\@endtheorem{\qed\endtrivlist\@endpefalse } % insert `\qed` macro
\makeatother
\theoremstyle{definition}
\newtheorem{example}{Example}[chapter]

%% Utilities
% TODOs
\newcommand{\todo}[1]{{\par\noindent\small\color{red} \framebox{\parbox{\dimexpr\linewidth-2\fboxsep-2\fboxrule}{\textbf{TODO:} #1}}}}

% Convenient macros
\newcommand{\linksify}[1]{\texttt{#1}}
\newcommand{\defas}[0]{\mathrel{\overset{\makebox[0pt]{\mbox{\normalfont\tiny\sffamily def}}}{=}}} % "defined-as-equal"
\newcommand{\rulesep}{\unskip\ \vrule\ } % Inserts a vertical line

% Formalisation
%\newcommand{\Calc}{\ensuremath{\lambda_{\text{effrow}}}\xspace}
\newcommand{\Calc}{\ensuremath{\lambda_{\text{eff}}^\rho}\xspace}

\newcommand{\slab}[1]{\textrm{#1}}
\newcommand{\semlab}[1]{\text{\scshape{S-#1}}}
\newcommand{\tylab}[1]{\text{\scshape{T-#1}}}
\newcommand{\mlab}[1]{\text{\scshape{M-#1}}}
\newcommand{\klab}[1]{#1}
\newcommand{\revto}{\ensuremath{\leftarrow}}

\newcommand{\keyw}[1]{\textbf{#1}}
\newcommand{\Handle}{\keyw{handle}}
\newcommand{\With}{\keyw{with}}
\newcommand{\Let}{\keyw{let}}
\newcommand{\In}{\keyw{in}}
\newcommand{\Do}{\keyw{do}}
\newcommand{\Return}{\keyw{return}}
\newcommand{\Case}{\keyw{case}}
\newcommand{\Absurd}{\keyw{absurd}}
\newcommand{\Record}[1]{\ensuremath{\langle #1 \rangle}}

\newcommand{\Pre}[1]{\mathsf{Pre}(#1)}
\newcommand{\Abs}{\mathsf{Abs}}
\newcommand{\Presence}{\mathsf{Presence}}
\newcommand{\Row}{\mathsf{Row}}
\newcommand{\Type}{\mathsf{Type}}

\newcommand{\Int}{\mathsf{Int}}
\newcommand{\Bool}{\mathsf{Bool}}

\newcommand{\True}{\mathsf{true}}
\newcommand{\False}{\mathsf{false}}

\newcommand{\cek}[1]{\ensuremath{\langle #1 \rangle}}

% configurations
\newcommand{\conf}{\mathcal{C}}

% typing
%% \newcommand{\eff}{\mathbin{!}}
\newcommand{\eff}{!}
\newcommand{\typ}[2]{#1 \vdash #2}
\newcommand{\typv}[2]{#1 \vdash #2}
\newcommand{\typc}[3]{#1 \vdash #2 \eff #3}

\newcommand{\harrow}[2]{\ensuremath{\mathbin{~^{#1}\!\!\Rightarrow^{#2}}}}
%\newcommand{\Harrow}[4]{#1 \harrow{#2}{#4} #3}
\newcommand{\Harrow}[4]{#1!#2 \Rightarrow #3!#4}

% stacks
\newcommand{\nil}{\ensuremath{[\,]}}
\newcommand{\cons}{\ensuremath{::}}

% These operators are taken from Conor's Agda course:
%   * fish appends a cons list onto a snoc list by reversing the cons list
%     (yielding a snoc list)
%   * chips appends a cons list onto a snoc list by reversing the snoc list
%     (yielding a cons list)
%   * chips is revapp (assuming we represent snoc lists as cons lists)
\newcommand{\fish}{\ensuremath{\mathbin{<\!>\!\!<}}}
\newcommand{\chips}{\ensuremath{\mathbin{<\!>\!\!>}}}

%\newcommand{\revapp}{\ensuremath{\mathbin{<\!\!\!+\!\!+}}}
\newcommand{\revapp}{\chips}

\newcommand{\concat}{\mathbin{+\!\!+}}


%% Effects on the turnstyle
%\newcommand{\typ}[3]{#1 \vdash_{#3} #2}


% environments
\newcommand{\env}{\gamma}

\newcommand{\reducesto}[0]{\ensuremath{\leadsto}}
\newcommand{\stepsto}[0]{\ensuremath{\longrightarrow}}
\newcommand{\Stepsto}{\Longrightarrow}

% array stuff
\newcommand{\ba}{\begin{array}}
\newcommand{\ea}{\end{array}}

\newcommand{\bl}{\ba[t]{@{}l@{}}}
\newcommand{\el}{\ea}

% Syntax environment
\newenvironment{syntax}{\[\ba{@{}l@{\quad}r@{~}c@{~}l@{}}}{\ea\]\ignorespacesafterend}
\newenvironment{reductions}{\[\ba{@{}l@{\qquad}@{}r@{~~}c@{~~}l@{}}}{\ea\]\ignorespacesafterend}

\newenvironment{eqs}{\ba{@{}r@{~}c@{~}l@{}}}{\ea}
\newenvironment{equations}{\[\ba{@{}r@{~}c@{~}l@{}}}{\ea\]\ignorespacesafterend}


% translations
\newcommand{\val}[2]{\llbracket #1 \rrbracket #2}
\newcommand{\inv}[1]{\llparenthesis #1 \rrparenthesis}

% restrict an environment
\newcommand{\res}{\backslash}

%% Information about the title, etc.
\title{Compilation of Effect Handlers and their Applications in Concurrency}
%\title{Compiling Effect Handlers}
\author{Daniel Hillerström}

%% If the year of submission is not the current year, uncomment this line and 
%% specify it here:
\submityear{2016}

%% Specify the abstract here.
\abstract{%
  \todo{Refine abstract} 
%
  Applications are comprised of \emph{effectful operations}, such as
  raising exceptions, thread forking, writing and reading from files
  or updating mutable state.  Often effectful operations have a
  salient impact on the behaviour of an application. Yet in many
  programming models such operations subsist as uncontrollable
  side-effecting actions, that occur implicitly on ``the side'' during
  run-time.

%
Algebraic effects combined with effect handlers provide an interface for controlling effectful operations.

% 
  Many programs comprise effectful operations such as thread forking,
  raising exceptions, writing and reading from files or updating some
  mutable state.  have a salient impact on the behaviour of programs.
  During run-time programs may perform several side-effecting
  actions. Effects are a salient part of Programs Algebraic effects
  combined with effect handlers provide a modular abstraction for
  effectful programming.
%
Concurrency and parallelism 
%%
The key insight is to classify handlers according to their linearity.
%%

We present a core calculus \Calc{} with row-polymorphic effects and affine handlers based on a variation of A-normal form used in our implementation. In addition, we give an operational semantics for the calculus, which we use to prove the soundness of the calculus.
}

%% Now we start with the actual document.
\begin{document}
\raggedbottom
%% First, the preliminary pages
\begin{preliminary}

%% This creates the title page
\maketitle

%% Acknowledgements
\begin{acknowledgements}
Acknowledgements will appear here\dots

Thanks Christophe\dots

Thanks Sam\dots

Thanks Office 1.07\dots

Thanks OCaml Labs and KC\dots

Thanks to Caoimhín Laoide-Kemp\dots

This work was supported in part by the EPSRC Centre for Doctoral Training in Pervasive Parallelism, funded by the UK Engineering and Physical Sciences Research Council (grant EP/L01503X/1) and the University of Edinburgh.
\end{acknowledgements}

%% Next we need to have the declaration.
\standarddeclaration

%% Finally, a dedication (this is optional -- uncomment the following line if
%% you want one).
%\dedication{To my mummy.}

\begin{preface}
A preface will possibly appear here\dots
\end{preface}

%% Create the table of contents
\setcounter{secnumdepth}{2} % Numbering on sections and subsections
\setcounter{tocdepth}{3} % Show chapters, sections and subsections in TOC
%\singlespace
\tableofcontents
%\doublespace

%% If you want a list of figures or tables, uncomment the appropriate line(s)
% \listoffigures
% \listoftables
\end{preliminary}

%%%%%%%%%%%%%%%%%%%%%%%%%%%%%%%%%%%
%%          Main content         %%
%%%%%%%%%%%%%%%%%%%%%%%%%%%%%%%%%%%

%%
%% Introduction
%%
\chapter{Introduction}
\label{ch:introduction}

For more than a decade the \emph{free lunch} has been over
\citep{Sutter2005}. Free lunch is a euphemism for the precursory
improvement in software application performance due to the rapid
growth of processor clock frequencies. However, the growth has
stagnated due to physical limitations \citep{Sutter2005}.
%
Nowadays we have \emph{multicore} processors. A multicore processor
comprises multiple computational cores that execute
simultaneously. Unfortunately, software applications must be designed
to take advantage of the additional processing power offered by
multicore processors. As a consequence we no longer get our lunch for
free.

%

% For more than a decade the \emph{free lunch} has been over
% \citep{Sutter2005}. The saying ``free lunch'' refers to the regular
% increase in performance, that software developers enjoyed without
% having to rewrite their applications, due to the regular increase in
% processing power as processor clock frequencies doubled every other
% year.

%

With the emergence of multicore processors parallelism has become
pervasive. As a result we are faced with the challenge of
\emph{parallel programming}. The aim of parallel programming is to
maximise program efficiency by utilising multiple computational cores
simultaneously.
%
An essential component of parallel programming is decomposition of a
program into tasks that can be run simultaneously on multiple
cores. Those tasks must then be scheduled across the available
computational resources.  Thread and process schedulers govern how the
computational resources of multicore processors are deployed. However,
compilers for most mainstream programming languages come with a
definitive scheduler that is hard wired into a complex, monolithic
runtime system.


% parallel programming is well-poised to become a dominant
% programming discipline that every software developer ought to know and
% master. But, conventional programming language are ill-suited for the
% challenge poised by parallel programming.

%

% While we continue to enjoy an increase in processing power, we do not
% get have our lunch for free anymore as clock frequencies no longer
% rise, which is mainly due to physical issues such as excessive heat,
% power consumption, and current leakage problems
% \citep{Sutter2005}. Instead a single computational device now comprise
% multiple processing cores. Consequently, we are faced with the
% challenge of \emph{parallel programming}, i.e.  programming multiple
% processing cores to execute simultaneously. Existing programs must be
% rewritten in order to take advantage of multiple cores, and hence,
% performance no longer increases for free.

%

In this work we adopt an approach akin to Multicore OCaml
\citep{Dolan2014,Dolan2015} by using \emph{handlers for algebraic
  effects} \citep{Plotkin2001,Plotkin2003,Plotkin2013} to enable
programmers to write user-level, modular schedulers along the lines of
\citet{KC2016}. Our story is twofold: we study novel applications of
effect handlers in parallelism, as well as presenting a novel
compilation technique for effect handlers.

%

% In this work, we shall consider \emph{handlers for algebraic effects}
% \citep{Plotkin2001,Plotkin2003,Plotkin2013} as an approach to
% concurrent and parallel programming. Our story is twofold: We study
% novel process-oriented applications of effect handlers in concurrent
% and parallel programming, but we also present a novel compilation
% technique for effect handlers.

\section{Problem analysis}

\todo{Parallelism is multiplicity of processors simultaneously}
%
\todo{Unit of concurrency: Fibers} \todo{Unit of parallelism: Domains}
%
\todo{Don't conflate concurrency and parallelism; context switching
  brings death to scaling}
%
\todo{Map fibers onto domains ($\#domains \approx \#cores$)}
Parallelism is simultaneous execution
%
of computations. Concurrency is overlapped execution of processes.
Parallelism is about utilising multiple computational resources
simultaneously to maximise efficiency. The aim is speed up a
computation by delegating different parts of it to different
processors to execute at the same time \citep{Marlow2013}.

By contrast, concurrency is concerned with structuring interactions
between multiple agents. Conceptually, 

% In theory,
% parallelism is easy, since running any two sequential programs in
% parallel have the same semantics as running them sequentially. In
% practice, though, parallelism is often hard because we have far more
% computations to delegate than we have available resources, thus we
% must carefully \emph{schedule} computations.


\section{Problem statement}
\section{Aims and objectives}
Our main contributions are
\begin{itemize}
  \item A compiler for Links with effect handlers, that supports the
    full abstraction provided by deep handlers making novel use of the
    existing linear type system of Links to guide the code generator for
    handlers.
  \item A reconstruction of the message-passing concurrency model of
    Links in terms of effect handlers. 
  \item Novel applications of effect handlers in parallelism.
\end{itemize}
\section{Thesis outline}
\section{Typographical conventions}

%%
%% Background
%%

\chapter{Links, OCaml, and effect handlers}
\label{ch:background}
This chapter introduces algebraic effects and their
handlers. Specifically, in Section~\ref{sec:theory-handlers}
introduces the theory, while Sections~\ref{sec:links-handlers} and
\ref{sec:ocaml-handlers} give an introduction to programming with
effect handlers in Links and OCaml, respectively.

\section{A theory of algebraic effects and handlers}
\label{sec:theory-handlers}

\section{Links primer}
\label{sec:links-primer}

We omit the implementation of the process queue data
structure. Instead we describe its interface:
\begin{itemize}
\item We assume a type constructor \lstinline$Queue(a)$ that
  constructs a queue type whose elements have type \lstinline$a$.
\item We assume a function
  \lstinline$enqueue : (Queue(a), a) -> Queue(a)$ that given a queue
  and an element appends the element onto a copy of the queue.
\item Similarly, the function
  \lstinline$dequeue : (Queue(a)) ~> (a, Queue(a))$ given a queue it
  returns the head and a copy of the tail of the queue. An attempt to
  dequeue the empty queue is considered an error.
\end{itemize}

\section{Links with effect handlers}
\label{sec:links-handlers}

\begin{figure}
  \centering
  \begin{sequencediagram}
    \newthread{A}{Program}{}
    \newinst[5]{B}{maybeResult}{}
    \newinst[2.5]{C}{randomResult}{}
    \newinst[0.5]{D}{drunkToss}{}
    \begin{call}[2]
      {A}{\lstinline$maybeResult(randomResult(drunkToss))$}{B}{\lstinline$Just(Tails)$}
      \begin{call}[3]
        {B}{\lstinline$randomResult(drunkToss)()$}{C}{\lstinline$Tails$}
        \begin{call}[2]
          {C}{\lstinline$drunkToss()$}{D}{\lstinline$Tails$}
          \begin{call}[2]
            {D}{\lstinline$do Choose$}{C}{\lstinline$k(true)$}
          \end{call}
          \begin{call}[2]
            {D}{\lstinline$do Choose$}{C}{\lstinline$k(false)$}
          \end{call}
        \end{call}
      \end{call}
    \end{call}

  \end{sequencediagram}
  \caption{Sequence diagram.}\label{fig:sequence}
\end{figure}

\subsection{Structural typing}

\subsection{Affine and multi-shot effect handlers}
\label{sec:links-affine-multi}

This section provides a short primer to effect handlers in Links. The
contents of this section are largely based on
\cite{Hillerstrom2016b}. An algebraic effect is given by a signature
of \emph{abstract operations}. For example \emph{nondeterminism} is an
algebraic effect that is given by a nondeterministic choice operation
called \lstinline$Choose$. In Links, we may use this operation to
implement a coin toss:
%
\snippet{toss.links}
%
This declares an \emph{abstract computation} \lstinline$toss$, which
invokes an operation \lstinline$Choose$ using the \lstinline$do$
primitive.  The \lstinline$sig$ keyword begins a signature, which
reads: \lstinline$toss$ is a computation with effect signature
\lstinline${Choose:Bool |e}$ and return value \lstinline$Toss$, whose
constructors are \lstinline$Heads$ and \lstinline$Tails$.  Links
employs row typing to support extensible effect signatures, thus
\lstinline$e$ is an effect variable, which can be instantiated with
additional operations.

Introduction of another operation causes the effect signature to grow
accordingly. For example, if we introduce an exception operation
\lstinline$Fail : Zero$, then we can model a drunk coin toss:
%
\snippet{drunkToss.links}
%
Here \lstinline$Zero$ is the empty type, and thus the
\lstinline$switch$ pattern matching construct has no clauses.

An effect handler instantiates a subset of the operations of an
abstract computation. For example, the following handler interprets
\lstinline$Choose$ randomly:
%
\snippet{randomResult.links}
%
The signature conveys that the handler interprets the operation
\lstinline$Choose$ and leaves any other operations uninterpreted. The
notation \lstinline$Choose{_}$ denotes that the operation is
polymorphic in its presence.  The handler comprises two clauses:
\begin{enumerate}[1)]
  \item the \lstinline$Return$-clause specifies how to handle the return
    value of the computation.
  \item the \lstinline$Choose$-clause specifies how to handle a
    \lstinline$Choose$ operation. The parameter \lstinline$k$ is the
    (delimited) continuation of the operation \lstinline$Choose$ in the
    computation.
\end{enumerate}
We say that \lstinline$randomResult$ is a \emph{linear handler},
because it invokes every continuation exactly once. One possible
output of \lstinline$randomResult(toss)$ is
%
\lstinputlisting{outputs/randomResult.output}
%
Essentially, the handler interprets the computation \lstinline$toss$
as modelling a \emph{fair} coin, that is the outcomes ought to be
uniformly distributed.

Alternatively, we may define a handler for \lstinline$Choose$ that
invokes its continuation twice to enumerate every possible outcome:
%
\snippet{allResults.links}
%
Observe that the return value is lifted into a singleton list. The
\lstinline$Choose$-clause concatenates the outcomes obtained by
interpreting the operation as \lstinline$true$ and \lstinline$false$,
respectively. We say that \lstinline$allResults$ is a \emph{multi-shot
  handler}. The result obtained by \lstinline$allResults(toss)$ is
%
\lstinputlisting{outputs/allResults.output}
%

Finally, we have handlers that do not invoke continuations. These are
familiar \emph{exception handlers}. As an example consider the
following handler, which returns \lstinline$Just$ the result of the
computation or returns \lstinline$Nothing$ if the operation
\lstinline$Fail$ is performed:
%
\snippet{maybeResult.links}
%
The type system prevents invocation of the continuation in the
\lstinline$Fail$-clause, because the type \lstinline$Zero$ has zero
inhabitants. Linear and exception handlers together constitute
\emph{affine handlers}.

\section{OCaml with handlers}
\label{sec:ocaml-handlers}

\subsection{Nominal typing}

%%
%% Related work
%%
\chapter{Related work}
\label{ch:related-work}

\section{Implementations of effect handlers}
Any signature of abstract operations can be understood as a free
algebra and represented as a functor. In particular, every such
functor gives rise to a free monad. Thus, free monads provide a
natural basis for implementing effect handlers (\citet{Swierstra2008b}
provide an account of free monads for functional programmers).  Many
of the library implementations of effect handlers include
implementations based on free monads~\citep{Kammar2013, Kiselyov2013,
  Kiselyov2015, Brady2013, Wu2014}.

\citet{Kammar2013} provide an implementation of effect handlers using
a continuation monad, which completely avoids materialising any data
constructors. \citet{Wu2015} explain how it works, by taking advantage
of Haskell's fusion optimisations. This approach does appear to depend
rather critically on the handlers being deep rather than shallow, and
in Haskell it relies on them being type classes, and hence not really
first class.

The Idris effects library by \cite{Brady2013} takes advantage of
dependent types to provide effect handlers for a form of effects
corresponding to parameterised monads~\citep{Atkey09}.
%
In the effects library, effects are represented as lists of types.

We are aware of three languages that are specifically designed with
effect handlers in mind.
%
\begin{itemize}
\item The Eff language by \cite{Bauer2015} is a strict language with
  Hindley-Milner type inference similar in spirit to ML, but extended
  with effect handlers.
%
It includes a novel feature for supporting fresh generation of effects
in order to support effects such as ML-style higher-order state (which
has an operation for generating new references).
%
The original version of Eff~\citep{Bauer2015} does not include an
effect type system. However, an effect type system has subsequently
been experimented with~\citep{BauerP13, Pretnar2014}.
%
This effect type system is considerably more complicated than ours. It
makes essential use of subtyping, includes a region system, and a form
of effect polymorphism, which one might reasonably cast as a form of
row polymorphism.

\item Frank by \cite{McBride2014} takes the idea of effect handlers to
  the extreme, having no primitive notion of function, only
  handlers. In Frank a function is but a special case of a handler.
  Frank is built on a bidirectional type system. It includes an effect
  type system and a novel form of effect polymorphism in which the
  programmer never needs to read or write any effect
  variables. Frank's effect system can be viewed as implementing a
  form of row polymorphism. Unlike Links, but much like
  Koka by \cite{Leijen14}, Frank allows multiple occurrences of the same
  label in a row. In contrast rows in Links are based on Remy's design
  in which duplicates are not allowed, but negative information is.

\item Shonky by \cite{McBride2016} amounts to a dynamically-typed variant
  of Frank. Though it is not statically typed, handlers must be
  annotated with the names of the effects that they handle. The
  implementation of Shonky is quite similar to ours in that it uses a
  generalisation of the CEK machine. The main differences are that
  Shonky does not use an ANF representation, so has more forms of
  continuation to handle, and in contrast to our nested continuation
  structure, Shonky uses a completely flat structure.
  % and where our continuations have a nested
  % structure, Shonky uses a completely flat structure for
  % continuations.
\end{itemize}

Although OCaml itself has no support for effect handlers, a
development branch, Multicore OCaml~\cite{Dolan2015}, does. Multicore
OCaml does not include an effect type system, and handlers are
restricted so that continuations are affine, that is, they can be
invoked at most once. This design admits a particularly efficient
implementation, as continuations need never be copied, so they can
simply be stored on the stack.

\section{Parallel programming models}
\todo{\cite{Marlow2013}}

%%
%% Concurrent programming with handlers
%%
\chapter{Encoding concurrency models with handlers}
\label{ch:programming}

In this chapter we explore the admissibility of building an entire
concurrency model with effect handlers. As a starting point we attempt
to reconstruct the built-in concurrency model of Links, therefore we
begin with a tour of the Links model in
Section~\ref{sec:links-model}. Afterwards in
Section~\ref{sec:links-model-handlers} we demonstrate a possible
encoding of the concurrency model using algebraic effects and
handlers. In the subsequent Sections~\ref{sec:thread-model} and
\ref{sec:???} we explore further developments of the model.

\section{The built-in concurrency model of Links}
\label{sec:links-model}

The concurrency model of Links is based on a typed \emph{actor} model
\citet{Cooper2006}. In an actor model processes run in (memory)
isolation. A process can only make local state changing decisions
directly. In order to influence the global program state, the process
must communicate with other processes through message passing. Each
process is equipped with its own mailbox -- which may be either
\emph{buffered} or \emph{unbuffered}. In Links mailboxes are buffered,
that is a mailbox is a collection of messages that can be consumed one
at a time. 
%
\todo{Tie this together somewhat better}
Figure~\ref{fig:links-model} displays an abstract representation of
the interaction between two processes. The process $P_1$ cannot access
the state of process $P_2$ -- and vice versa. Each mailbox is
intrinsic to each process. The process $P_1$ interact with $P_2$ via a
message $s$ depending on the contents of $s$ the process $P_2$ may
choose to alter its own state.
%
\begin{figure}
\centering
\begin{tikzpicture}[node distance=5pt,every node/.style={draw, outer sep=0pt}]
\tikzset{sarrow/.style={-latex,thick,dotted}}


  \node(P0) [draw=none,minimum width=30pt]         {$P_1$};
  \node(P0M0) [left=of P0,minimum height=22pt] { $m_0$ };
  \node(P0M1) [left=of P0M0,xshift=5pt,minimum height=22pt,minimum width=22pt] { $m_1$ };
  \node(P0MC) [left=of P0M1,xshift=5pt,minimum height=22pt,minimum width=22pt] { $\cdots$ };
  \node(P0MN) [left=of P0MC,xshift=5pt,minimum height=22pt,minimum width=22pt] { $m_k$ };
  \node(P0MT) [left=of P0MN,draw=none] { \textit{\textbf{\footnotesize{Mailbox}}} };
  \node(P0M) [fill=black,fill opacity=0.05,fit={(P0M0) (P0M1) (P0MC) (P0MN) (P0MT)}] { };
  \node(P0B) [fit={(P0M) (P0)}] { };

  \node(P1) [draw=none,below=of P0,yshift=-70pt,minimum width=30pt]         {$P_2$};
  \node(P1M0) [left=of P1,minimum height=22pt] { $n_0$ };
  \node(P1M1) [left=of P1M0,xshift=5pt,minimum height=22pt,minimum width=22pt] { $n_1$ };
  \node(P1MC) [left=of P1M1,xshift=5pt,minimum height=22pt,minimum width=22pt] { $\cdots$ };
  \node(P1MN) [left=of P1MC,xshift=5pt,minimum height=22pt,minimum width=22pt] { $s$ };
  \node(P1MT) [left=of P1MN,draw=none] { \textit{\textbf{\footnotesize{Mailbox}}} };
  \node(P1M) [fill=black,fill opacity=0.05,fit={(P1M0) (P1M1) (P1MC) (P1MN) (P1MT)}] { };
  \node(P1B) [fit={(P1M) (P1)}] { };
  
%  \draw[sarrow] (P0B.east) to node[draw=none,midway,right,align=center,out=40,in=] { asd } (P1B.west);
%\path[every node/.style={font=\sffamily\small}]
%    (P0B.south east) edge [out=300, in=120,style={-latex,thick,dotted}]  (P1B.north west);
\draw[sarrow] (P0B.south east) to [out=300, in=120,style={-latex,thick,dotted}]  (P1B.north west) node [draw=none,midway,yshift=-35pt,xshift=-80pt] {$P_1$ sends message $s$ to $P_2$};
\end{tikzpicture}
\caption{Interaction between processes.}
\label{fig:links-model}
\end{figure}

A Links program begins in a single thread of control but can fork into
multiple processes. There are four essential built-in primitives for
concurrency: process \emph{self referral}, process \emph{spawning},
\emph{sending} and \emph{receiving} messages. In the following
paragraphs we briefly introduce each primitive.

\paragraph{Self referral} The built-in function \lstinline$self$
retrieves the process identifier from the current context. A process
identifier has the type \lstinline$Process({ |e })$. The type is
parameterised by an effect row with an effect variable \lstinline$e$
which tracks the effects that the process may perform. The signature
of \lstinline$self$ is
\begin{lstlisting}
links> self;
self : () ~e~> Process ({ |e })
\end{lstlisting}
%
Invoking the function at the top level retrieves the identifier of the
main process:
%
\begin{lstlisting}
links> self();
0 : Process ({ |_ })
\end{lstlisting}
%
Evidently the main process always has identifier
\lstinline$0$. Subsequent processes are assigned identifiers
\lstinline$1$, \lstinline$2$, \lstinline$3$, and so forth. Although, the term
\lstinline$0$ looks like a value of the integer type we cannot act
upon it as such, because the process type is implemented as an
abstract type. Therefore the type checker will prevent us from
incrementing the process identifer by hand:
%
\begin{lstlisting}
links> self() + 1;
<stdin>:1: Type error: [..]
\end{lstlisting}
%
Here, we have omitted the full error message for brevity, however the
problem is that the addition operator \lstinline$(+)$ expects two
arguments of type \lstinline$Int$. The two types
\lstinline$Process({ |e })$ and \lstinline$Int$ are incompatible. This
adds a layer of safety to the concurrency model as we cannot
erroneously refer to an non-existent process.
%

\paragraph{Spawning} The primitive \lstinline$spawn { expression }$
returns a handle to a new process which begins by evaluating
\lstinline$expression$. For example by spawning the computation
\lstinline$print("Hello World")$ we obtain the process identifier for the subprocess:
%
\begin{lstlisting}
links> spawn { print("Hello World") };
Hello World
1 : Process ({ wild|_ })
\end{lstlisting}
%
We obtain the handle \lstinline$1$ whose effect row contains the
\lstinline$wild$ effect due to the fact that the subprocess prints to
the standard out.

\paragraph{Receiving and sending} A process can receive messages using
the \lstinline$recv$ function, e.g.
%
\begin{lstlisting}
links> var p1 = spawn { print("Message: " ^^ recv()) };
p1 = 1 : Process ({ hear:String,wild|_ })
\end{lstlisting}
%
The \lstinline$recv$ function blocks until a message becomes
available.  Each mailbox is given a static type according to the
messages it expects to receive. The built-in effect \lstinline$hear$
reflects and tracks this type.
%
Now we can send message to process \lstinline$1$ using the \lstinline
$(!)$ (pronounced ``send''):
%
\begin{lstlisting}
links> p1 ! "Hello";
Message: Hello
() : ()
\end{lstlisting}
%
Process \lstinline$1$ prints the received message \lstinline$"Hello"$
and terminates afterwards. The process is only capable of receiving
strings but often a process will use a variant to tag the different
messages it can receive. Typically, a process will dispatch on the tag
of received message. As a concrete example consider a coffee machine process that 
gets informed about passing comets and celebrity sightings:
\begin{lstlisting}
var p3 = spawn {
  fun loop() {
    var _ = switch(recv()) {
      case PassingComet(id, zenith, azimuth) -> cometSighted(id, zenith, azimuth)
      case CelebritySighting(name, venue)    -> celebSighted(name, venue)
    };
    loop()
  }
  loop()
};
\end{lstlisting}
%
Here we assume the existence of two functions \lstinline$cometSighted$
and \lstinline$celebSighted$ that register sightings of comets and
sightings of celebrities, respectively. The \lstinline$hear$ effect in
the type signature of \lstinline$p3$ now reflects that the process
expects to receive messages tagged by either
\lstinline$CelebritySighting$ or \lstinline$PassingComet$:
%
\begin{lstlisting}
links> p3;
3 : Process ({ hear:[|CelebritySighting:(String, String)
                     |PassingComet:(Int, Float, Float)|]
             , wild|_ })
\end{lstlisting}
Now, we can inform the process of any passing comets and celebrity sightings:
%
\begin{lstlisting}
links> p3 ! CelebritySighting("Ewan McGregor", "Leith");
Ewan McGregor has been seen in Leith
() : ()
links> p3 ! PassingComet(42, 10.3, 180.5);
Comet no. 42 sighted (10.3, 180.5)
() : ()
\end{lstlisting}
%

\subsection{Sieve of Eratosthenes example}
\label{sec:sieve-example}

We will now consider a larger example which will serve to demonstrate
an actual concurrent application in Links. However, we shall reuse the
example to demonstrate our reconstructed concurrency model in
Section~\ref{sec:links-model-handlers} too.

We shall implement a parallel version of the \emph{Sieve of
  Eratosthenes} prime number finding algorithm. 
% %Starting from a prime
% number (e.g. $2$) the sequential version of the algorithm iteratively
% marks multiples of the given prime number as composite up to some
% specified limit. Thereafter the algorithm repeats the entire procedure
% on the first unmarked marker, which must be a prime. The algorithm
% continues so until some specified limit.

\begin{figure}
\tikzset{my ellipse/.style={
        draw=black, 
        ultra thick, 
        ellipse, 
        anchor=west,
        minimum width=70pt,
        minimum height=35pt
        },
}
\tikzset{my arrow/.style={-latex,thick}}
\tikzset{sarrow/.style={-latex,thick,dotted}}

\centering
\begin{tikzpicture}[decoration=snake]

\node(P0)[my ellipse] {$P_1$ (Generator)};

\node(P1)[my ellipse,below=of P0,yshift=-20pt] {$P_2^2$};
\node(P2)[my ellipse,right=of P1,xshift=85pt] {$P_3^3$};
\node(P3)[my ellipse,below=of P2,yshift=-20pt] {$P_4^5$};
\node(P4)[my ellipse,left=of P3,xshift=-85pt] {$P_5^7$};

\draw[sarrow] (P0) to node[midway,right,align=center,yshift=2pt ] { $\{2,\dots,10\}$ } (P1);
\draw[sarrow] (P1) to node[midway,above,align=center,yshift=2pt ] { $\{n \mid n \not\equiv 0 \; (\text{mod } 2) \}$ } (P2);
\draw[sarrow] (P2) to node[midway,left,align=center ] { $\{n \mid n \not\equiv 0 \; (\text{mod } 3)\}$ } (P3);
\draw[sarrow] (P3) to node[midway,below,align=center,yshift=-2pt ] { $\{n \mid n \not\equiv 0 \; (\text{mod } 5)\}$ } (P4);

\end{tikzpicture}
\caption{Visual representation of Sieve of Eratosthenes.}\label{fig:sieve}
\end{figure}

The parallel version constructs a pipeline of processes where each
process holds one prime number
\citep{Andrews2000}. Figure~\ref{fig:sieve} visualises the sieve
pipeline which finds primes between $2$ and $10$. In the figure the
subscript of each process is its identifier and the superscript is the
prime number it holds. The job of each sieve process is to receive and
perform a primality test on candidate prime by dividing the candidate
number by its own prime. If the remainder after division is positive
then the process forwards the candidate to its neighbour process. The
initial process generates a sequence of natural numbers. These numbers
are sent one by one to the first sieve process. We implement this as
the function \lstinline$generator(n)$ where \lstinline$n$ is the upper
bound on sequence being generated:
%
\begin{lstlisting}
fun generator(n) {
  var first = spawnAngel { sieve() };
  foreach([2..n], fun(p) { first ! Candidate(p) });
  first ! Stop
}
\end{lstlisting}
%
The function spawns the first sieve as an angel process. The
\lstinline$foreach$ function has type
\lstinline$([a], (a) ~e~> ()) ~e~> ()$, that is it takes a list and an
action as arguments. The action is applied to each element in the
list. The notation \lstinline$[2..n]$ is a shorthand for generating
the sequence of integers between $2$ and $n$. The action function
sends a \lstinline$Candidate$-tagged number to the first sieve
process. When the entire sequence has been transmitted the generator
sends the \lstinline$Stop$ signal.

The first message sent to a sieve process will always be its
prime. Subsequent messages may either be a \lstinline$Candidate$ prime
number or the \lstinline$Stop$ signal. The implementation of
\lstinline$sieve$ is given below.
%
\begin{lstlisting}[numbers=left]
fun sieve() {
  var myprime = fromCandidate(recv());
  print(intToString(myprime));
  fun loop(neighbour) {
    switch (recv()) {
       case Stop -> stop(neighbour)
       case Candidate(number) ->
       if (number `mod` myprime == 0) {
          loop(neighbour)
       } else {
          var neighbour =
            switch (neighbour) {
              case Just(pid) -> pid
              case Nothing   -> spawnAngel { sieve() }
            };
          neighbour ! Candidate(number);
          loop(Just(neighbour))
       }
    }
  }
  loop(Nothing)
}
\end{lstlisting}
%
We describe function line by line.
\begin{description}
\item[Line 2] receives the process's prime number. The
  \lstinline$fromCandidate$ simply removes the \lstinline$Candidate$
  tag from the number.
\item[Line 3] simply prints the prime to standard out.
\item[Line 4] begins the definition of process's main loop. The
  function is parameterised by a handle to its neighbouring process.
\item[Lines 5-9] dispatch on the message tag. If the message is
  \lstinline$Stop$ then the auxiliary function \lstinline$stop$
  (described below) propagates the stop signal to the neighbour
  process. If a candidate prime number is received then the process
  performs a primality test on the candidate number. In case the
  number is composite the \lstinline$loop$ function recursively calls
  itself to repeat the procedure.
\item[Lines 11-17] forwards the candidate \lstinline$number$ to its
  neighbour. However before doing so the process must ensure it has a
  neighbour. If the process already has a neighbour then the
  \lstinline$switch$ expression simply removes the \lstinline$Just$
  tag from the neighbour's identifier. In case it does not have a
  neighbour it spawns one and returns the new neighbour's identifier.
  Thereafter the process rewraps the candidate number and sends it to
  its neighbour. Finally, the \lstinline$loop$ function gets called
  recursively with the neighbour's identifier wrapped in a
  \lstinline$Just$.
\end{description}
%
The \lstinline$stop$ function handles the special case of when the
process has no neighbour:
%
\begin{lstlisting}
fun stop(neighbour) {
  switch (neighbour) {
    case Nothing   -> ()
    case Just(pid) -> pid ! Stop
  }
}
\end{lstlisting}
%
The process silently exits if it does not have a neighbour. Otherwise
the process forwards the \lstinline$Stop$ message before exiting. Now,
we can run the example:
\begin{lstlisting}
links> generator(10);
2
3
5
7
() : ()
\end{lstlisting}
%
As expected it the primes between $2$ and $10$ get printed.

\section{A reconstruction of the Links concurrency model with handlers}
\label{sec:links-model-handlers}

We will now attempt to reconstruct the concurrency of model of Links
using effect handlers. We will do this reconstruction in several steps:
first we consider how to design the process abstraction. Afterwards we
consider the problem of scheduling processes. Thereafter we consider
how to implement communication between processes.

\subsection{Process abstraction}
\label{sec:links-model-handlers-process}

Initially, we choose to represent process identifiers as integers:
%
\snippet{pidInt.links}
%
As for the process type we parameterise it by the effects that it may
perform, thus we name it \lstinline$EProcess$ for \emph{effectful
  process}:
%
\snippet{eprocess.links}
%
The type is a simple alias for the singleton record type with an
\lstinline$id$ field of type \lstinline$Pid$. However, the process
type is parameterised by a \emph{phantom} row \lstinline$e$ -- it is
not mentioned in the definition of \lstinline$EProcess$ -- which we
will use to track the effects of processes. We use a small trick to
get effect tracking to work as desired. The trick is obvious: let the
type and effect system do the work for us. We make essential use of a
helper function to create processes:
%
\snippet{makeProcess.links}
%
The trick lies in the type signature. First note that the function
accepts as its first argument a nullary function that has effect row
\lstinline$e$ and returns unit. We intent that this nullary function
is the computation that the process will execute. Now, observe that
the function wholly ignores the first argument. It only uses the
second argument to construct a singleton record is structural
compatible with an inhabitant of the type \lstinline$EProcess$. The
type system captures the effect row on the first argument and uses it
to construct a process type with an open row that mentions
\lstinline$e$, i.e. \lstinline$EProcess({ |e})$. Because the process
type mentions \lstinline$e$ it must, by row polymorphism, have the
same effects as the ignored input function. We can try it out to see
that it indeed works:
%
\begin{lstlisting}
links> makeProcess(fun() { print("I am wild!") }, 1);
(id=1) : EProcess ({ |wild|_ })
links> makeProcess(fun() { if (do Choose) print("True") 
......                     else print("False") }, 2);
(id=2) : EProcess ({ |Choose:Bool,wild|_ })
\end{lstlisting}
%

\subsection{Spawning, suspending, and scheduling processes}
\label{sec:links-model-handlers-scheduler}

\todo{Introduce fibers}
Though, the caveat is that cooperative multi-tasking does not
immediately provide preemptive concurrency. 

\subsubsection{Concurrency operations}

We require an interface to spawn and suspend processes. We define an
operation \lstinline$Spawn$ that takes a nullary function with an
effect signature \lstinline$e$ as its argument and returns a process
of type \lstinline$EProcess({ |e})$, i.e.
%
\snippet{pspawn.links}
%
We dub the function \lstinline$pspawn$ for \emph{process} spawn. As
usual the operation is quite simple. Most of the work is concerned
with getting the type signature right. The signature enforces any
interpretation of \lstinline$pspawn$ to at least return a process
handle. In a similar fashion we define an operation for suspending a
computation:
%
\snippet{yield.links}
%
This settles the interface for fibers. Next, we need to give a
concrete implementation of fibers through an interpretation of
\lstinline$Spawn$ and \lstinline$Yield$.

\subsubsection{A basic round-robin scheduler}

An interpretation of \lstinline$Spawn$ and \lstinline$Yield$ amounts
to scheduling processes. Our programs will be running in the a single
thread of control and therefore it is the responsibility of the
scheduler to share execution time among processes.

We will begin by considering a simple fair, round-robin scheduler. The
main idea is to maintain a process queue with first-in-first-out
(FIFO) semantics.  There are two obvious scheduling policies to choose
from when a process invokes the \lstinline$Spawn$ operation. The
scheduler either enqueues the parent process and runs the child
process immediately or enqueues the child process and resumes the
parent process. We will adopt the former policy. Using this policy an
invocation of \lstinline$Yield$ simply enqueues the yielding process
and dequeues the next process to run. We implement the scheduler as a
handler:
%
\lstinputlisting[numbers=left]{snippets/basicRoundrobin.links}
%
We describe the handler line by line.
%
\begin{description}
\item[Lines 2-5] handle the \lstinline$Spawn$ operation by first
  creating an process handle of type
  \lstinline[mathescape]!EProcess($\cdot$)! for the new
  \lstinline$child$ process. Note that for now we impetuously assign
  the identifier \lstinline$0$ to every process. Later we will rectify
  this. The handle is returned to the parent via the resumption
  function, \lstinline$resume$. However, the evaluation of the
  resumption function is delayed as we store it inside a thunk.  This
  effectively amounts to suspending the process. The thunk goes into
  the process queue. Afterwards the scheduler transfers control to the
  forked computation \lstinline$f$. It is crucial that the scheduler
  is invoked recursively in order to handle any effects that
  \lstinline$f$ may perform.

\item[Lines 6-8] suspend the yielding process and transfers control to
  the next process in the queue. The \lstinline$dequeueProcess$
  returns a thunk which starts evaluating immediately.

\item[Lines 9-10] handle the case when a process terminates. This
  leaves room for another process to run hence we dequeue the next
  process.

\end{description}
%

The process queue is inherently stateful as it changes every time a
process is spawned or suspended.  The auxiliary functions
\lstinline$enqueueProcess$ and \lstinline$dequeueProcess$ use the
abstract state operations \lstinline$Get$ and \lstinline$Put$ that we
introduced in Section~\ref{sec:??} to maintain the queue. The
underlying queue data structure is the one we implemented in
Section~\ref{sec:links-primer}. The \lstinline$enqueueProcess$ simply
retrieves and updates the queue, i.e.
%
\snippet{enqueueProcess.links}
%
Similarly, \lstinline$dequeueProcess$ retrieves and removes a process
from the queue. However if the queue is empty then it returns the
trivial process. This way the scheduler does not need to handle any
special cases. The implementation is straightforward:
%
\snippet{dequeueProcess.links}
%
Now we have all the necessary infrastructure in place to spawn simple
non-interacting processes. In the following small example we spawn a
family of non-interacting processes where each generation has only one
child:
%
\snippet{spawnLoneChild.links}
%
To run this example we plug our handlers together, i.e.
%
\begin{lstlisting}
run -<- evalState(emptyQueue()) 
    -<- basicRoundrobin -< spawnLoneChild(2);
\end{lstlisting}
%
Initially the process queue is empty hence why we seed the state
handler with the empty queue. The program compiles and runs:
%
\begin{lstlisting}[style=terminal]
$ ./links -c basic_concur_model.links -o basic_model
$ ./basic_model
Spawned P#0
Spawned P#0
\end{lstlisting}
%
Under the \lstinline$basicRoundrobin$ scheduler a parent and its child
process are indistinguishable because the scheduler assigns both of
them the \lstinline$0$ identifier.

\subsubsection{Unique processes}

The scheduler \lstinline$basicRoundrobin$ neatly demonstrates that the
essence of the scheduling is fairly simple: variations on enqueuing
and dequeuing of processes. However, in order to have interacting
processes we need to be able to uniquely identify processes. 

As a first attempt at making processes unique we might consider a
parameterised version of basic round-robin scheduler that carries the
identifier of the executing process, that is
%
\snippet{wrongRoundrobin.links}
%
%This scheduler follows the same pattern as \lstinline$basicRoundrobin$
The handler takes a parameter \lstinline$pid$ which is supposed to be
identifier of the executing process. When the executing process spawns
a new child process that child process gets assigned the identifier
\lstinline$pid+1$. The parent process gets suspended before control is
transferred to the child process. 

Since the handler is parameterised the resumption function
\lstinline$resume$ becomes a curried function that carries over the
values of the handler parameter to a subsequent invocation of the
handler. An invocation of \lstinline$resume(child)(pid)$ orchestrates
the initial return of control to the parent process. First the process
handle \lstinline$child$ gets passed to the parent process. Afterwards
it installs parent identifier \lstinline$pid$ as the identifier of the
executing process.

Let us try out this scheduler on the previous example. We plug
everything together as before, except that the scheduler now takes the
identifier of the root process, which we will define to be \lstinline$0$:
%
\begin{lstlisting}
run -<- evalState(emptyQueue()) 
    -<- wrongRoundrobin(0) -< spawnLoneChild(2);
\end{lstlisting}
%
Compiling and running the program yields a promising
result:
%
\begin{lstlisting}[style=terminal]
$ ./links -c wrong_concur_model1.links -o wrong_model1
$ ./wrong_model1
Spawned P#1
Spawned P#2
\end{lstlisting}
Whilst this is the result we desire the model is not correct. The following example illustrates the problem with the model:
%
\snippet{spawnSiblings.links}
%
The difference from the previous program \lstinline$spawnLoneChild$ is
that each child gets a sibling. Substituting \lstinline$spawnSiblings$
for \lstinline$spawnLoneChild$ above yields:
%
\begin{lstlisting}[style=terminal]
$ ./links -c wrong_concur_model2.links -o wrong_model2
$ ./wrong_model2
Spawned P#1
Spawned P#2
Spawned P#1
Spawned P#2
Spawned P#2
Spawned P#2
\end{lstlisting}
%
This is definitely not what we wanted. The scheduler
\lstinline$wrongRoundrobin$ fails to capture the hierarchical
structure properly.
\todo{Condense the above to just a few lines? Change process identifier to process name?}

In order to obtain a robust unique identifier generation scheme we
have to detach the identifier generation from the state of the
executing process. We can achieve this by turning identifier
generation into an abstract operation then we can give it a stateful
interpretation. Thus we introduce an operation \lstinline$FreshName$
that generates a new name (or identifier) of type \lstinline$a$:
%
\snippet{freshname.links}
%
We implement a function \lstinline$names$ that generates a suitable
name generator. The function takes as arguments a generating function
and a seed value:
%
\snippet{freshnames.links}
%
The generated handler \lstinline$h$ is parameterised by the next fresh
name. Observe that the handler essentially interprets the
\lstinline$FreshName$ operation as performing both state operations
\lstinline$Get$ and \lstinline$Put$. First the \lstinline$name$ is
retrieved afterwards a new name is generated by
\lstinline$gen(name)$. We can use \lstinline$names$ to implement a
process identifier generator:
%
\snippet{pidgenerator.links}
%
Now, we have to update our round-robin scheduler to use the name
generation operation. However, simply using \lstinline$freshName$ in
the \lstinline$Spawn$-clause will not be enough as the initial process
is spawned through the handler. Thus the initial process would be
nameless. The workaround is to define a function
\lstinline$upRoundrobin$ (``up'' for \emph{unique processes}) that
embeds a scheduler. Prior to invoking the scheduler the function
generates a name for initial process:
%
\lstinputlisting[linebackgroundcolor={\lstcolorlines{1,2,5,14}}]{snippets/uniqueRoundrobin.links}
%
The implementation of \lstinline$scheduler$ is similar to
\lstinline$basicRoundrobin$ the only difference is the use of
\lstinline$freshName$. The first invocation of \lstinline$freshName$
generates a name for the initial process which is the input
computation \lstinline$m$. We run the scheduler on the initial process
to handle its effects. We end up with the following pipeline of
handlers:
%
\begin{lstlisting}
run -<- pidgenerator 
    -<- evalState(emptyQueue()) -<- upRoundrobin 
    -<  spawnSiblings(2);
\end{lstlisting}
%
Compiling and running the program yields:
%
\begin{lstlisting}[style=terminal]
$ ./links -c up_concur_model.links -o up_model
$ ./up_model
Spawned P#1
Spawned P#2
Spawned P#3
Spawned P#4
Spawned P#5
Spawned P#6
\end{lstlisting}
%
As we see every process gets a unique name.

\subsubsection{Self-referral}

We now consider a final refinement of the round-robin scheduler:
process self-referral. We introduce a new operation \lstinline$Myself$:
%
\snippet{myself.links}
%
The idea is that an invocation of \lstinline$Myself$ should return the
handle of the calling process. We may implement this functionality
through a small extension to our scheduler. The main idea is to let
the scheduler keep track of the executing process. One possible way to
enable this tracking is to parameterise the scheduler by the executing
process:
%
\lstinputlisting[linebackgroundcolor={\lstcolorlines{3,7,8,10,12,13}},numbers=left]{snippets/roundrobin.links}
%
We describe the changes line by line.
\begin{description}
\item[Line 3] begins the definition of \lstinline$scheduler$ that now
  takes a parameter \lstinline$activeProcess$ which is a handle to the
  current executing process.
\item[Lines 7-8] suspends the parent process. As usual the
  \lstinline$resume$ function returns the child handle. In addition it
  also sets the executing process. Once \lstinline$resume$ gets
  invoked control gets transferred back to the parent process, hence
  the parent process becomes the active process.  The
  \lstinline$scheduler$ is invoked with \lstinline$child$ as the
  active process.
\item[Line 10] suspends the executing process and wakes up the next
  process to run.
\item[Lines 12-13] returns the handle to the executing process. This
  clause, unlike the two previous clauses, does not alter the state of
  the executing process.
\end{description}

To illustrate the extension in action consider a variation of the
\lstinline$spawnSiblings$ example:
%
\lstinputlisting[linebackgroundcolor{\lstcolorlines{2}}]{snippets/spawnFamily.links}
%
In this example each process announces that it has been spawned rather
than its parent. Thus, we do not care about the process names returned
by the calls to \lstinline$pspawn$. We wire everything together as
follows:
%
\begin{lstlisting}
run -<- pidgenerator 
    -<- evalState(emptyQueue()) -<- roundrobin 
    -<  spawnFamily(2);
\end{lstlisting}
%
Compiling and running the program produce the desired result:
%
\begin{lstlisting}[style=terminal]
$ ./links -c rr_concur_model -o rr_model
$ ./rr_model
Spawned P#0
Spawned P#1
Spawned P#2
Spawned P#3
Spawned P#4
Spawned P#5
Spawned P#6
\end{lstlisting}
%
From the perspective of computations the initial process is no longer
a special case. Every process gets handled uniformly by this
scheduler. 

\subsection{Handling communication}
\label{sec:links-model-handlers-communication}

In this section we will augment the implementation of the concurrency
model with primitives for interaction between processes. Our
concurrency model is an instance of a message-passing model, thus we
need at least a primitive for sending messages and another for
receiving them.

\subsubsection{Sending and receiving}

The built-in concurrency model of Links implements both blocking send
and receive. We name it \lstinline$psend$ (for \emph{process} send):
%
\snippet{psend.links}
%

We define an infix operator that looks similar to the bang operator
(\lstinline$!$):
%
\snippet{psendOp.links}
%

%
\snippet{precv.links}
%
This code implements a blocking receive.  The \lstinline$loop$
function takes a process handle as its input and keeps looping until a
message has been received. To obtain the process handle we invoke
\lstinline$Myself$ operation. The \lstinline$loop$-body invokes the
operation \lstinline$Recv$ which returns a \lstinline$Maybe$-value. If
there are no messages then the function invokes itself in order to
retry. However, before retrying the function yields in order to let
another process run.

\subsection{Sieve of Eratosthenes example revisited}
\label{sec:links-model-handlers-example}

Generator
%
\snippet{generator.links}
%
Sieve
%
\snippet{sieve.links}
%
Now we can plug everything together
%
\begin{lstlisting}
run -<- pidgenerator
    -<- evalState(emptyDictionary()) -<- mailbox 
    -<- evalState(emptyQueue())      -<- roundrobin 
    -<  generator(10);
\end{lstlisting}
Next, we can compile the program and run it:
\begin{lstlisting}
$ ./links -c concur_model.links -o concur_model
$ ./concur_model
2
3
5
7
\end{lstlisting}
% \section{Pipeline parallelism}
% \label{sec:prime}

% \section{Unbalanced parallelism}
% \begin{figure}
% \centering
% \begin{lstlisting}[style={},basicstyle=\ttfamily\tiny,breaklines=false,morekeywords={\#},keywordstyle={\color{blue}}]
% ....................................................................................................
% ....................................................................................................
% .................................................................##.................................
% .................................................................##.................................
% ...............................................................#####................................
% ..............................................................#######...............................
% ..............................................................#######...............................
% .............................................................########...............................
% ..............................................................#######...............................
% ..............................................................#######...............................
% ..............................................................######................................
% ............................................................#....#.#................................
% ........................................................##.##############..#........................
% .................................................##.....####################........................
% .................................................###..########################...#.#................
% ................................................###############################.###.................
% ..................................................##################################................
% .................................................##################################.................
% ................................................#################################...................
% ................................................##################################..................
% .............................................#.####################################.................
% ..............................................######################################................
% .............................................#######################################................
% ...........................................############################################.............
% ............................................###########################################.............
% ...........................................##########################################...............
% ........................#.....#...........###########################################...............
% ........................##..####.#........############################################..............
% ........................############......#############################################.............
% ........................#############.....############################################..............
% .......................###############...#############################################..............
% ......................#################..#############################################..............
% ......................#################..############################################...............
% .....................###################.############################################...............
% .....................###################.###########################################................
% .................###.###################.###########################################................
% ................########################.#########################################..................
% #################################################################################...................
% ................########################.#########################################..................
% .................###.###################.###########################################................
% .....................###################.###########################################................
% .....................###################.############################################...............
% ......................#################..############################################...............
% ......................#################..#############################################..............
% .......................###############...#############################################..............
% ........................#############.....############################################..............
% ........................############......#############################################.............
% ........................##..####.#........############################################..............
% ........................#.....#...........###########################################...............
% ...........................................##########################################...............
% ............................................###########################################.............
% ...........................................############################################.............
% .............................................#######################################................
% ..............................................######################################................
% .............................................#.####################################.................
% ................................................##################################..................
% ................................................#################################...................
% .................................................##################################.................
% ..................................................##################################................
% ................................................###############################.###.................
% .................................................###..########################...#.#................
% .................................................##.....####################........................
% ........................................................##.##############..#........................
% ............................................................#....#.#................................
% ..............................................................######................................
% ..............................................................#######...............................
% ..............................................................#######...............................
% .............................................................########...............................
% ..............................................................#######...............................
% ..............................................................#######...............................
% ...............................................................#####................................
% .................................................................##.................................
% .................................................................##.................................
% ....................................................................................................
% ....................................................................................................
% \end{lstlisting}
% \caption{Mandelbrot set.}\label{fig:mandelbrot}
% \end{figure}

% \begin{figure}[t!]
% \begin{subfigure}[t]{0.5\textwidth}
% \centering
% \begin{tikzpicture}
%   \begin{axis}[ 
%     xlabel=$x$,
%     ylabel={$f(x) = \cosh^4(x)$},
%     ylabel near ticks,
%     enlargelimits=0.1,
%     xmin=-5,xmax=5
%   ] 
%     \addplot[name path=f,domain=-5:5] {cosh(x) * cosh(x) * cosh(x) * cosh(x)}; 
%   \end{axis}
% \end{tikzpicture}
% \caption{Plot of $f(x) = \cosh^4(x)$.}
% \end{subfigure}
% \hspace{1.0cm}
% \begin{subfigure}[t]{0.5\textwidth}
% \centering
% \begin{tikzpicture}
%   \begin{axis}[ 
%     xlabel=$x$,
%     ylabel={$f(x) = \cosh^4(x)$},
%     ylabel near ticks,
%     enlargelimits=0.1
%   ] 
%     \addplot[name path=f,domain=0:5] {cosh(x) * cosh(x) * cosh(x) * cosh(x)}; 

%     \path[name path=axis] (axis cs:0,0) -- (axis cs:5,0);

%     \addplot [
%         thick,
%         color=blue,
%         fill=blue, 
%         fill opacity=0.08
%     ]
%     fill between[
%         of=f and axis,
%         soft clip={domain=0:5},
%     ];
%   \end{axis}
% \end{tikzpicture}
% \caption{The integral $\displaystyle\int_0^5 \cosh^4(x)\,  \mathop{\mathrm{d}x}$.}
% \label{fig:integral}
% \end{subfigure}
% \caption{}\label{fig:quad}
% \end{figure}

%%
%% Compiling handlers
%%
\chapter{Compiling effect handlers}

\section{Translating Links ANF into Lambda}
\label{sec:translation}

\begin{figure}
\tikzset{my rectangle/.style={
        draw=black,
        thick, 
        rectangle, 
        minimum width={width("OCaml frontend")+2pt}}
}
\centering
\begin{tikzpicture}[node distance=5pt,every node/.style={draw, minimum width=100pt, outer sep=0pt}]

  \node(A)          {OCaml frontend};
  \node(B)[right=of A,xshift=40pt] {Links frontend};

  \node(C)[inner sep=0pt,yshift=-50pt,fit={(A) (B)},label=center:Lambda] {};

  \node[above=of C,draw=none,yshift=-2pt] {\footnotesize{\textit{\textbf{OCaml backend}}}};

  \node(D)[yshift=-90pt]          {Byte code};
  \node(E)[right=of D,xshift=40pt] {Flambda};

  \node(G)[below=of E,yshift=-10pt] {Clambda};
  \node(H)[below=of G,yshift=-10pt] {Native backends};

  \node(F)[fit=(G)(H),inner sep=0,left=of G.north west,anchor=north east,xshift=-40pt]          {Custom backends};


  \node(Z)[fit=(C)(H),inner sep=20pt] {};

  \draw[->] (A.south) to ([xshift=50pt]C.north west);
  \draw[->] (B.south) to ([xshift=-50pt]C.north east);

  \draw[->] ([xshift=-72pt]C.south) to (D.north);

  \draw[->] (D) to (F);

  \draw[->] ([xshift=72pt]C.south) to (E.north);
  \draw[->] (E) to (G);
  \draw[->] (G) to (H);
%  \node(C)[below=of B] {C};
%  \node(D)[below=of C] {D};

%  \node(Z)[fit=(A)(D),right=of A.north east,anchor=north west, inner sep=0] {Z};

% % Frontends
% \node [my rectangle] (OCaml frontend) at (0,0) {OCaml frontend};
% \node [my rectangle] (Links frontend) at (6,0) {Links frontend};

% % OCaml backend
% \node [my rectangle,minimum width=270pt] (Lambda) at (3,-2) {Lambda};
% %\node[rectangle,draw=black,ultra thick,minimum height=+5.0cm,minimum width=+4.0cm,fit ={(Parser.north) (Typechecker.south)}] (Frontend) {};
\end{tikzpicture}
\caption{OCaml backend}\label{fig:infra-diagram}
\end{figure}

We reuse most of the previous Links infrastructure. The Links frontend
is type-checked and translated into a small, typed intermediate
language in \emph{A-normal form} (ANF) \citep{Flanagan1993}. The Links
interpreter implements a generalised CEK machine
\citep{Hillerstrom2016a}, which interprets ANF code.

%

Our compilation strategy is to translate the Links ANF language into the OCaml
\emph{Lambda} language, which is a small, untyped lambda calculus. The OCaml
backend exposes a hierarchy of intermediate representations (IRs), where the
top representation is known as Lambda. As shown in the Figure
\ref{fig:infra-diagram}, the Lambda IR offers two different compilation
options: byte code and native code. Therefore by targeting Lambda rather than
a lower level IR, we achieve maximum flexibility as a translation into byte
code, in principle, enables us to take advantage of custom backends such as
\texttt{js\_of\_ocaml} to produce efficient JavaScript.

%

There are several semantic differences between Links and OCaml, e.g. Links
employs structural typing, whilst OCaml predominantly employs nominal typing.
In particular, Links employs row typing for effects, records, and variants,
whereas OCaml only supports row typing for the latter. Exhibiting a faithful
translation from Links to OCaml amounts to a lot of value boxing. Thus, we
target Lambda for greater flexibility and control.  We effectively subvert
OCaml's typechecker by targeting Lambda, however the translation is safe as
Links programs are already typechecked.

\begin{figure}
\centering
\begin{tikzpicture}
  \drawstruct{(4.5,2)}
  \structcell{\lstinline$do Choose$}  \coordinate (DoChoose) at (currentcell.west);
  \structcell{$\cdots$}
  \structcell{\lstinline$do Fail$} \coordinate (DoFail) at (currentcell.east);
  \structcell{\lstinline$Return-clause$}
  \structcell{\lstinline[keywordstyle=\footnotesize]$Exception handler$}
  \structcell{\lstinline$Choose-clause$} \coordinate (HandleChoose1) at (currentcell.west); \coordinate (HandleChoose2) at (currentcell.east);
  \structcell{$\cdot$} \coordinate (H1) at (currentcell.west);

  \drawstruct{(0,0)}
  \structcell{$\cdots$} \coordinate (H2) at (currentcell.east);
  \structcell{\lstinline$Return-clause$}
  \structcell{\lstinline[keywordstyle=\footnotesize]$Exception handler$}
  \structcell{\lstinline$Fail-clause$}
  \structcell{$\bot$}

%  \draw[<-] (H2) .. controls ([yshift=0cm] H2) and ([xshift=-10cm,yshift=4cm] H1) .. (H1);
  \draw (H2) edge[out=-20,in=-175,<-] (H1);
  \draw (DoChoose) edge[out=175,in=-180,->] (HandleChoose1);
  \draw (DoFail) edge[xshift=0.5cm,out=0,in=-360,->] (HandleChoose2);
\end{tikzpicture}
\caption{Representation of \lstinline$maybeResult(randomResult(drunkToss))$ at runtime.}\label{fig:rtstack}
\end{figure}

\section{Runtime Representation}
By using the OCaml backend we naturally inherit the OCaml
run-time. OCaml implements effect handlers as heap-managed stack data
structures, and as a consequence composition of handlers gives rise to
$n$-element stacks at run-time. For example, the composition
\lstinline[mathescape]!randomResult(maybeResult($\cdot$))! is
represented as a two-element stack. Thus, an invocation of an abstract
operation amounts to performing a dynamic lookup for a suitable
handler in a stack.

Since the primary use of handlers in OCaml is to express concurrency,
OCaml handlers are affine; continuations can only be resumed at most
once, and multiple invocations of a continuation causes a run-time
error. Multi-shot handlers can be simulated by manually cloning
continuations using \lstinline$Obj.clone_continuation$. The cost of
cloning is linear in the size of the handler stack. However, cloning
is a fragile abstraction; if the handler stack contains at least one
multi-shot handler, then every affine handler in the stack must be
demoted to a multi-shot handler to be safe, because a multi-shot
handler may consume a linear continuation more than
once. Consequently, multi-shot handlers in OCaml break modularity.

In the Links compiler we use the cloning capability under the hood to implement
multi-shot handlers. For example, our encoding of \lstinline$allResults$
amounts to the following in plain OCaml:
% \kc{$Obj.clone$ was too generic and I plan to use $Obj.clone\_continuation$ to
% be less ambiguous.}
\begin{lstlisting}[style=ocaml]
let all_results m = match m () with
 | x -> [x]
 | effect Choose k -> 
   let k' arg = 
     continue (Obj.clone_continuation k) arg 
   in k' true @ k' false
\end{lstlisting}
OCaml provides a unified syntax for pattern-matching on regular, effect, and
exception patterns. The keyword \lstinline[style=ocaml]$effect$ begins an
operation-clause. Essentially, we create a local function \lstinline$k'$, which
wraps the actual continuation \lstinline$k$. An invocation of \lstinline$k'$
passes its argument to a fresh copy of the actual continuation. The
\lstinline$continue$ function is provided by the standard library; given a
continuation and a value, it invokes the continuation with that particular
value. By default we implement every handler as a multi-shot handler.

%%
%% Evaluation
%%
\chapter{Experiments}
\label{ch:experiments}

\section{Methodology}
\label{sec:methodology}

%%
%% Formalisation
%%
\chapter{A calculus of handlers and rows}
\label{sec:lambe-eff-row}

In this section, we present a type and effect system and a small-step
operational semantics for \Calc (pronounced ``lambda-eff-row''), a
Church-style row-polymorphic call-by-value calculus for effect
handlers.
%
This core calculus captures the essence of the Links IR.
%
We prove that the operational semantics is sound with respect to the
type and effect system.

A key advantage of row polymorphism is that it integrates rather
smoothly with Hindley-Milner type inference. We concern ourselves only
with the explicitly-typed core language, as the treatment of type
inference is quite standard.

The design of \Calc is inspired by the $\lambda$-calculi of
\citet{Kammar2013}, \citet{Pretnar2015}, and \citet{Lindley2012}.
%
As in the work of \citet{Kammar2013}, each handler can have its own
effect signature. As in the work of \citet{Pretnar2015}, the
underlying formalism is fine-grain call-by-value~\cite{LevyPT03},
which names each intermediate computation like in A-normal
form~\cite{Flanagan1993}, but unlike A-normal form is closed under
$\beta$-reduction. As in the work of \citet{Lindley2012}, the effect
system is based on row polymorphism.

\section{Types}
The grammars of types, effects, kinds, and type and kind environments
are given in Figure~\ref{fig:types-syntax}.

\begin{figure}
Types
\begin{syntax}
\slab{Value types}    &A,B  &::= & %\Bool \mid \Int
                                      A \to C
                               \mid  \forall \alpha^K.C \\
                             &&\mid& \Record{R} \mid [R]
                               \mid  C \Rightarrow D \mid \alpha \\
                               %% \mid  \Harrow{A}{E}{B}{E'} \mid \alpha \\
\slab{Computation types} 
                      &C,D  &::= & A \eff E \\
\slab{Effect types}   &E    &::= & \{R\}\\
\slab{Row types}      &R    &::= & \ell : P;R \mid \rho \mid \cdot \\
\slab{Presence types} &P    &::= & \Pre{A} \mid \Abs \mid \theta\\
%\slab{Labels}         &\ell &    &                \\
\slab{Kinds}          &K    &::= & \Type \mid \Row_\mathcal{L} \mid \Presence\\
\slab{Label sets}     &\mathcal{L} &::=& \emptyset \mid \{\ell\} \uplus \mathcal{L}\\
%\slab{Type variables} &\alpha, \rho, \theta& \\
\slab{Type environments} &\Gamma &::=& \cdot \mid \Gamma, x:A \\
\slab{Kind environments} &\Delta &::=& \cdot \mid \Delta, \alpha:K \\
\end{syntax}
\caption{Types, effects, kinds, and environments}
\label{fig:types-syntax}
\end{figure}

\paragraph{Value Types}
%The base types are integers and booleans.
The function type $A \to C$ takes an argument of type $A$ and returns
a computation of type $C$.
%% The effect signature enumerates the
%% operations that the function may perform during evaluation.
The polymorphic type $\forall \alpha^K .\, C$ is parameterised by a
type variable $\alpha$ of kind $K$. The record type $\Record{R}$
represents records with fields given by labels of row $R$. Dually, the
variant type $[R]$ represents a sum of fields tagged by the labels of
row $R$. The handler type $C \Rightarrow D$ transforms a computation
of type $C$ into a computation of type $D$.

\paragraph{Computation Types}
A computation type $A \eff E$ is given by a value type $A$ and an
effect $E$, which specifies the operations that the computation may
perform.

\paragraph{Row Types}
Effect types, records and variants are defined in terms of rows.
A row type embodies a collection of distinct labels, each of which is
annotated with a presence type. A presence type indicates whether a
label is \emph{present} with some type $A$ ($\Pre{A}$), \emph{absent}
($\Abs$) or \emph{polymorphic} in its presence ($\theta$).

Row types are either \emph{closed} or \emph{open}. A closed row type
ends in~$\cdot$, whilst an open row type ends with a \emph{row
  variable} $\rho$. Furthermore, a closed row term can have only the
labels explicitly mentioned in its type. Conversely, the row variable
in an open row can be instantiated with additional labels. We identify
rows up to reordering of labels, for instance, we consider the
following two rows equivalent:
\[ \ell_1 : P_1; \cdots; \ell_n : P_n \equiv \ell_n : P_n; \cdots ; \ell_1 : P_1. \]
The unit and empty type are definable in terms of row types. We define
the unit type as the empty, closed record, that is,
$\Record{\cdot}$. Similarly, we define the empty type as the empty,
closed variant $[\cdot]$. Usually, we usually omit the $\cdot$ for
closed rows.

\paragraph{Kinds}
We have three kinds: $Type$, $\emph{Row}_\mathcal{L}$ and $Presence$
which classify value types, row types and presence types,
respectively. Row kinds are annotated with a set of labels
$\mathcal{L}$. The kind of a complete row is
$Row_{\mathcal{\emptyset}}$. More generally, the kind
$Row_{\mathcal{L}}$ denotes a partial row which cannot mention the
labels in $\mathcal{L}$.
%

\paragraph{Type Variables}
We let $\alpha$, $\rho$ and $\theta$ range over type variables. By
convention we use $\alpha$ for value type variables or for type
variables of unspecified kind, $\rho$ for type variables of row kind,
and $\theta$ for type variables of presence kind.

\paragraph{Type and Kind Environments}
Type environments map term variables to their types and kind
environments map type variables to their kinds.

\section{Terms}
\begin{figure}
\begin{syntax}
                             % SL: no need for constants in the core calculus
%\slab{Constants}     &c    &::= & n \mid \True \mid \False \\
\slab{Values}        &V,W  &::= & x
                             %\mid c
                             \mid \lambda x^A .\, M \mid \Lambda \alpha^K .\, M  \\
                     &     &\mid& \Record{} \mid \Record{\ell = V;W} \mid (\ell\, V)^R \\
                     &     &    &\\
\slab{Computations}  &M,N  &::= & V\,W \mid V\,A\\
                     &     &\mid& \Let\; \Record{\ell=x;y} \revto V \; \In \; N\\
                     &     &\mid& \Case\; V \{\ell\; x \mapsto M; y \mapsto N\} \mid \Absurd^C V\\
                     &     &\mid& \Return\; V \\
                     &     &\mid& \Let \; x \revto M \; \In \; N\\
                     &     &\mid& (\Do \; \ell \; V)^E \\
                     &     &\mid& \Handle \; M \; \With \; H\\
                     &     &    &\\
\slab{Handlers}      &H    &::= & \{ \Return \; x \mapsto M \} \\
                     &     &\mid& \{ \ell \; x \; k \mapsto M \} \uplus H \\
%% \slab{Integers}      &n    &      \\
%% \slab{Variables}     &x,y,z,k &    & \\
\end{syntax}

\caption{Term Syntax}
\label{fig:term-syntax}
\end{figure}
The terms are given in Figure~\ref{fig:term-syntax}. We let $x,y,z,k$
range over term variables. By convention, we use $k$ to denote
continuation names. %We let $n$ range over integer constants.

The syntax partitions terms into values, computations and
handlers. 
%
Value terms comprise variables ($x$), %constants ($c$),
lambda abstraction ($\lambda x^A . \, M$), type abstraction ($\Lambda
\alpha^K . \, M$), and the introduction forms for records and
variants. Records are introduced using the empty record $\Record{}$
and record extension $\Record{\ell = V; W}$, whilst variants are
introduced using injection $(\ell\, V)^R$ which injects a field with
label $\ell$ and value $V$ into a row whose type is $R$. We include
the row type annotation in order to support bottom-up type
reconstruction.

All elimination forms are computation terms. Abstraction and type
abstraction are eliminated using application ($V\,W$) and type
application ($V\,A$) respectively.
%
The record eliminator $(\Let \; \Record{\ell=x;y} \revto V \; \In \;
N)$ splits a record $V$ into $x$, the value associated with $\ell$,
and $y$, the rest of the record. Non-empty variants are eliminated
using the case construct ($\Case\; V\; \{\ell\; x \mapsto M; y \mapsto
N\}$), which evaluates the computation $M$ if the tag of $V$ matches
$\ell$, otherwise it falls through to $y$ and evaluates $N$.  The
elimination form for empty variants is ($\Absurd^C V$). A trivial
computation $(\Return\;V)$ returns value $V$. The expression $(\Let \;
x \revto M \; \In \; N)$ evaluates $M$ and binds the result value to
$x$ in $N$.

The construct $(\Do \; \ell \; V)^E$ invokes an operation $\ell$ with
value argument $V$. The handle construct $(\Handle \; M \; \With \;
H)$ runs a computation $M$ with handler definition $H$. A handler
definition $H$ consists of a return clause $\Return \; x \mapsto M$
and a possibly empty set of operation clauses $\{\ell_i \; x_i \; k_i
\mapsto M_i\}_i$. The return clause defines how to handle the final
return value of the handled computation, which is bound to $x$ in $M$.
%
The $i$-th operation clause binds the operation parameter to $x_i$ and
a the continuation $k_i$ in $M_i$.

We write $Id(M)$ for $\Handle \;M\; \With \; \{\Return\;x \mapsto
x\}$.
%
We write $H(\Return)$ for the return clause of $H$ and $H(\ell)$ for
the set of either zero or one operation clauses in $H$ that handle the
operation $\ell$. We write $dom(H)$ for the set of operations handled
by $H$.
%
As our calculus is Church-style, we annotate various term forms with
type or kind information (term abstraction, type abstraction,
injection, operations, and empty cases); we sometimes omit these
annotations.

%% SL: possibly a bit draconian, but I think we can do without the
%% detail here.

%% The terms of \Calc are written in a style reminiscent of
%% \emph{A-normal form} (ANF). ANF is a direct style intermediate
%% language in which every intermediate computation is let-bound
%% \cite{Flanagan1993}. As a consequence application consist of a value
%% applied to another value. For example, consider the Links expression
%% \lstinline$f(g(42))$, we must let bind the argument of \lstinline$f$
%% in order to bring the expression into ANF; we do not need to let bind
%% the argument of \lstinline$g$, since it is already a value, thus we
%% obtain the following ANF expression:
%% \[
%% \Let\; x \revto g\,42\; \In \; f\, x.
%% \]
%% However, ANF is not closed under $\beta$-reduction
%% \cite{Sabry1996,Kennedy2007}, since function application can lead to
%% nested \Let{}s. Consider the following example, which is adapted from
%% \citet{Sabry1996}:
%% \[ 
%%   \Let\; x \revto (\lambda y . \, \Let\; z \revto f \, a \;\In \; M) \, b \; \In\; N
%% \]
%% which $\beta$-reduces to 
%% \[
%%   \Let\; x \revto (\Let\; z \revto f \, a \;\In \; M) \; \In\; N
%% \]
%% which is not in ANF, because $x$ is bound to a computation rather than
%% a value. ANF can be recovered by adding a normalisation step
%% \cite{Sabry1996}.  We simplify our presentation by permitting nested
%% let-bindings and handlers to be applied directly to computations. The
%% resulting calculus is closer to fine-grained
%% call-by-value~\cite{LevyPT03}.

%% ,
%% therefore we say that \Calc-terms are in \emph{relaxed A-normal
%%   form} (RANF).

% As an example consider the translation of the Links expression \linksify{f(g(42),\{ var y = 1; h(y) \})} into its ANF. For clarity, we highlight the reducible term with a box:
% \begin{align*}
%             &  \fbox{\linksify{f(g(42),\{ var y = true; h(y) \})}}\\
% \Rightarrow &\quad\Let \; x_1 \revto g \; 42 \; \In\\
%             &\quad\quad\fbox{\linksify{f($x_1$,\{ var y = true; h(y) \})}}\\
% \Rightarrow &\quad\Let \; x_1 \revto g \; 42 \; \In\\
%             &\quad\quad\Let \; y \revto \Return \; \True \; \In\\
%             &\quad\quad\quad\fbox{\linksify{f($x_1$,h($y$))}}\\
% \Rightarrow &\quad\Let \; x_1 \revto g \; 42 \; \In\\
%             &\quad\quad\Let \; y \revto \Return \; \True \; \In\\
%             &\quad\quad\quad\Let \; x_2 \revto h \; y \; \In\\
%             &\quad\quad\quad\quad(f \; x_1)\; x_2
% \end{align*}
% Accordingly, the function \linksify{f} is applied only to value terms. Every subexpression gets decomposed into a single value gradually. As a consequence every subexpression has been explicitly named, thus in this respect, ANF is similar to the predominant imperative Static Single Assignment form.

\section{Static Semantics}
\label{sec:typing}
The kinding rules are given in Figure~\ref{fig:kinding} and the typing
rules are given in Figure~\ref{fig:typing}.

The kinding judgement $\Delta \vdash \alpha : K$ asserts that the type
variable $\alpha$ has kind $K$ in kind environment $\Delta$. The value
typing judgement $\typv{\Delta;\Gamma}{V : A}$ states that value term
$V$ has type $A$ under kind environment $\Delta$ and type environment
$\Gamma$. The computation typing judgement $\typc{\Delta;\Gamma}{M :
  A}{E}$ states that the term $M$ has type $A$ and effects $E$ under
kind environment $\Delta$ and type environment $\Gamma$. In typing
judgements, we implicitly assume that $\Gamma$, $E$ and $A$ are
well-kinded with respect to $\Delta$. We define the functions
$FTV(\Gamma)$ and $FTV(E)$ to be the set of free type variables in
$\Gamma$ and $E$, respectively.
%
%% We write $FTV(\Gamma,E)$ as shorthand for $FTV(\Gamma) \cup FTV(E)$.

The kind and typing rules are mostly straightforward. The interesting
typing rules are \tylab{Handle} and the two handler rules. The
\tylab{Handle} rule states that $\Handle\; M\; \With\; H$ produces a
computation of type $B$ given that the computation $M$ is typeable
under effect context $E$, and that $H$ is a handler which transforms a
computation of type $A$ with effect signature $E$ into another
computation of type $B$ with effect signature $E'$.

The \tylab{Handler} rule is crucial. The input effect $E$ and the
output effect $E'$ must share the same suffix $R$. This means that
$E'$ must explicitly mention each of the operations $\ell_i$, whether
that be to say that an $\ell_i$ is present with a given type
signature, absent, or polymorphic in its presence. The row $R$
describes the operations that are forwarded. It may include a
row-variable, in which case an arbitrary number of effects may be
forwarded by the handler.
%
The typing of the return clause is straightforward. In the typing of
each operation clause, the continuation returns the output computation
type $D$. Thus, we are here defining \emph{deep}
handlers~\cite{Kammar2013} in which the handler is implicitly wrapped
around the continuation, such that any subsequent operations are
handled uniformly by the same handler.
%
The Links implementation also supports \emph{shallow}
handlers~\cite{Kammar2013}, in which the continuation is instead
annotated with the input effect and one has to explicitly reinvoke the
handler after applying the continuation inside an operation clause.

%% \todo{Insert a forward reference to discussion of shallow handlers
%%   elsewhere.}

%% The \tylab{Deep} and \tylab{Shallow} rules for handlers describe how
%% to type deep and shallow handlers, respectively. The two rules are
%% similar, but differ crucially in their typing of the continuation
%% parameter $k$. The different typing is due to their different
%% behaviour. A deep handler wraps itself around continuations such that
%% any subsequent operations are handled uniformly. Therefore the return
%% type of a continuation type must match the input type of the
%% \Return{} case, and in particular, its effect signature must be the
%% same as the handler's output effect signature $E'$. Conversely, in a
%% shallow handler the continuation is annotated with the input effect
%% signature $E$ rather than the output effect signature. Moreover, the
%% return type $A$ is the same as return type of the handled
%% computation. Shallow handlers do not wrap themselves around the
%% continuation, therefore the continuation must be applied explicitly to
%% handler which has to handle the next operation
%% invocation. Consequently, operations may be handled non-uniformly.

%% For both handler types the input effect signature is a disjoint union
%% of operation signatures, that the handler directly interprets, and
%% some set $E_f$. The metavariable $E_f$ can be instantiated with a row
%% variable to include operations that are not explicitly mentioned by
%% the handler.

\begin{figure}
\begin{mathpar}
% alpha : K
  \inferrule*[Lab=\klab{TyVar}]
    { }
    {\Delta, \alpha : K \vdash \alpha : K}

% forall alpha : K . A : Type
  \inferrule*[Lab=\klab{Forall}]
    { \Delta, \alpha : K \vdash A : \Type \\
      \Delta, \alpha : K \vdash R : \Row_\emptyset}
    {\Delta \vdash (\forall \alpha^K . \, A \eff \{R\}) : \Type}
%%
%% % Int : Type
%%   \inferrule*[Lab=\klab{Int}]
%%     { }
%%     {\Delta \vdash \Int : \Type}
%%
%% % Bool : Type
%%   \inferrule*[Lab=\klab{Bool}]
%%     { }
%%     {\Delta \vdash \Bool : \Type}

% A -E-> B, A : Type, E : Row, B : Type
  \inferrule*[Lab=\klab{Fun}]
    { \Delta \vdash A : \Type \\
      \Delta \vdash R : \Row_\emptyset  \\
      \Delta \vdash B : \Type
    }
    {\Delta \vdash (A \to B \eff \{R\}) : \Type}

% Record
  \inferrule*[Lab=\klab{Record}]
    { \Delta \vdash R : \Row_\emptyset}
    {\Delta \vdash \Record{R} : \Type}

% Variant
  \inferrule*[Lab=\klab{Variant}]
    { \Delta \vdash R : \Row_\emptyset}
    {\Delta \vdash [R] : \Type}

% Present
  \inferrule*[Lab=\klab{Present}]
    {\Delta \vdash A : \Type}
    {\Delta \vdash \Pre{A} : \Presence}

% Absent
  \inferrule*[Lab=\klab{Absent}]
    { }
    {\Delta \vdash \Abs : \Presence}

% Empty row
  \inferrule*[Lab=\klab{EmptyRow}]
    { }
    {\Delta \vdash \cdot : \Row_\mathcal{L}}

% Extend row
  \inferrule*[Lab=\klab{ExtendRow}]
    { \Delta \vdash P : \Presence \\
      \Delta \vdash R : \Row_{\mathcal{L} \uplus \{\ell\}}
    }
    {\Delta \vdash \ell : P;R : \Row_\mathcal{L}}
\end{mathpar}

\caption{Kinding Rules}
\label{fig:kinding}
\end{figure}

\begin{figure*}
Values
\begin{mathpar}
% Variable
  \inferrule*[Lab=\tylab{Var}]
    {x : A \in \Gamma}
    {\typv{\Delta;\Gamma}{x : A}}
%%
%% % true : Bool
%%   \inferrule*[Lab=\tylab{True}]
%%     { }
%%     {\typv{\Delta;\Gamma}{\True : \Bool}}
%%
%% % false : Bool
%%   \inferrule*[Lab=\tylab{False}]
%%     { }
%%     {\typv{\Delta;\Gamma}{\False : \Bool}}
%%
%% % n : Int
%%   \inferrule*[Lab=\tylab{Int}]
%%     { n \in \mathbb{N} }
%%     {\typv{\Delta;\Gamma}{n : \Int}}

% Abstraction
  \inferrule*[Lab=\tylab{Lam}]
    {\typ{\Delta;\Gamma, x : A}{M : C}}
    {\typv{\Delta;\Gamma}{\lambda x^A .\, M : A \to C}}

% Polymorphic abstraction
  \inferrule*[Lab=\tylab{PolyLam}]
    {\typc{\Delta,\alpha : K;\Gamma}{M : A}{E} \\
     \alpha \notin FTV(\Gamma)
    }
    {\typv{\Delta;\Gamma}{\Lambda \alpha^K .\, M : \forall \alpha^K . \,A \eff E}}
\\
% unit : ()
  \inferrule*[Lab=\tylab{Unit}]
    { }
    {\typv{\Delta;\Gamma}{\Record{} : \Record{}}}

% Extension
  \inferrule*[Lab=\tylab{Extend}]
    { \typv{\Delta;\Gamma}{V : A} \\
      \typv{\Delta;\Gamma}{W : \Record{\ell:\Abs;R}}
    }
    {\typv{\Delta;\Gamma}{\Record{\ell=V;W} : \Record{\ell:\Pre{A};R}}}

% Inject
  \inferrule*[Lab=\tylab{Inject}]
    {\typv{\Delta;\Gamma}{V : A}}
    {\typv{\Delta;\Gamma}{(\ell\,V)^R : [\ell : \Pre{A}; R]}}
\end{mathpar}
Computations
\begin{mathpar}
% Application
  \inferrule*[Lab=\tylab{App}]
    {\typv{\Delta;\Gamma}{V : A \to C} \\
     \typv{\Delta;\Gamma}{W : B}
    }
    {\typ{\Delta;\Gamma}{V\,W : C}}

% Polymorphic application
  \inferrule*[Lab=\tylab{PolyApp}]
    {\typv{\Delta;\Gamma}{V : \forall \alpha^K . \, C} \\
     \Delta \vdash A : K
    }
    {\typ{\Delta;\Gamma}{V\,A : C[A/\alpha]}}

% Split
  \inferrule*[Lab=\tylab{Split}]
    {\typv{\Delta;\Gamma}{V : \Record{\ell : \Pre{A};R}} \\\\
     \typ{\Delta;\Gamma, x : A, y : \Record{\ell : \Abs; R}}{N : C}
    }
    {\typ{\Delta;\Gamma}{\Let \; \Record{\ell =x;y} \revto V\; \In \; N : C}}

% Case
  \inferrule*[Lab=\tylab{Case}]
    { \typv{\Delta;\Gamma}{V : [\ell : \Pre{A};R]}  \\\\
      \typ{\Delta;\Gamma,x:A}{M : C} \\\\
      \typ{\Delta;\Gamma,y:[\ell : \Abs;R]}{N : C}
    }
    {\typ{\Delta;\Gamma}{\Case \; V \{\ell\; x \mapsto M;y \mapsto N \} : C}}

% Absurd
  \inferrule*[Lab=\tylab{Absurd}]
    {\typv{\Delta;\Gamma}{V : []}}
    {\typ{\Delta;\Gamma}{\Absurd^C \; V : C}}

% Return
  \inferrule*[Lab=\tylab{Return}]
    {\typv{\Delta;\Gamma}{V : A}}
    {\typc{\Delta;\Gamma}{\Return \; V : A}{E}}

% Let 
  \inferrule*[Lab=\tylab{Let}]
    {\typc{\Delta;\Gamma}{M : A}{E} \\
     \typc{\Delta;\Gamma, x : A}{N : B}{E}
    }
    {\typc{\Delta;\Gamma}{\Let \; x \revto M\; \In \; N : B}{E}}
\\
% Do 
  \inferrule*[Lab=\tylab{Do}]
    {\typv{\Delta;\Gamma}{V : A} \\
     E = \{\ell : A \to B; R\}
    }
    {\typc{\Delta;\Gamma}{(\Do \; \ell \; V)^E : B}{E}}

% Open Handle
  \inferrule*[Lab=\tylab{Handle}]
    {\typv{\Delta;\Gamma}{M : C} \\
     \Delta;\Gamma \vdash H : C \Rightarrow D}
    {\typv{\Delta;\Gamma}{\Handle \; M \; \With \; H : D}}
\end{mathpar}

Handlers
\begin{mathpar}
% Closed handler
% \mprset{flushleft}
% \inferrule*[Lab=\tylab{Deep-Closed}]
%     { E = \{\ell_i : A_i \to B_i\}_i \\\\
%       H = \{\Return \; x \mapsto M\} \uplus \{\ell_i \; y_i \; k_i \mapsto N_i\}_i \\\\
%       \left[\Delta;\Gamma,y_i : A_i, k_i : B_i \xrightarrow{E'} C \vdash_{E'} N_i : C\right]_i \\\\
%       \Delta;\Gamma, x : A \vdash_{E'} M : C
%     }
%     {\Delta;\Gamma \vdash_{E'} H : (\Record{} \xrightarrow{E} A) \xrightarrow{E'} C}   

% Open handler
%\mprset{flushleft}
  \inferrule*[Lab=\tylab{Handler}]
    {
      C = A \eff \{(\ell_i : A_i \to B_i)_i; R\} \\
      D = B \eff \{(\ell_i : P_i)_i;         R\} \\
      H = \{\Return \; x \mapsto M\} \uplus \{\ell_i \; y \; k \mapsto N_i\}_i \\\\
      [\typv{\Delta;\Gamma,y : A_i, k : B_i \to D}{N_i : D}]_i \\
       \typv{\Delta;\Gamma, x : A}{M : D} \\
    }
    {{\Delta;\Gamma} \vdash {H : C \Rightarrow D}}

% Shallow closed handler
% \inferrule*[Lab=\tylab{Shallow-Closed}]
%     { E = \{\ell_i : A_i \to B_i\}_i \\\\
%       H = \{\Return \; x \mapsto M\} \uplus \{\ell_i \; y_i \; k_i \mapsto N_i\}_i \\\\
%       \left[\Delta;\Gamma,y_i : A_i, k_i : B_i \xrightarrow{E} A \vdash_{E'} N_i : C\right]_i \\\\
%       \Delta;\Gamma, x : A \vdash_{E'} M : C
%     }
%     {\Delta;\Gamma \vdash_{E'} H : (\Record{} \xrightarrow{E} A) \xrightarrow{E'} C}   

%% SL: we should mention this rule later

%% % Shallow open handler
%%   \inferrule*[Lab=\tylab{Shallow}]
%%     { E = \{\ell_i : A_i \to B_i\}_i \uplus E_f \\\\
%%       E' = E'' \uplus E_f \\\\
%%       H = \{\Return \; x \mapsto M\} \uplus \{\ell_i \; y \; k \mapsto N_i\}_i \\\\
%%       \left[\Delta;\Gamma,y : A_i, k : B_i \xrightarrow{E} A \vdash_{E'} N_i : C\right]_i \\\\
%%       \Delta;\Gamma, x : A \vdash_{E'} M : C
%%     }
%%     {\Delta;\Gamma \vdash_{E'} H : A \harrow{E}{E'} C}
\end{mathpar}

\caption{Typing Rules}
\label{fig:typing}
\end{figure*}

\section{Operational Semantics}
\label{sec:small-step}

\begin{figure*}

\begin{reductions}
\semlab{App}   & (\lambda x^A . \, M) V &\reducesto& M[V/x] \\
\semlab{TyApp} & (\Lambda \alpha^K . \, M) A &\reducesto& M[A/\alpha] \\
\semlab{Split} & \Let \; \Record{\ell = x;y} \revto \Record{\ell = V;W} \; \In \; N &\reducesto& N[V/x,W/y] \\
\semlab{Case$_1$} &
  \Case \; (\ell\, V)^R \{ \ell \; x \mapsto M; y \mapsto N\} &\reducesto& M[V/x] \\
\semlab{Case$_2$} &
  \Case \; (\ell\, V)^R \{ \ell' \; x \mapsto M; y \mapsto N\} &\reducesto& N[(\ell\, V)^R/y], \hfill\quad \text{if } \ell \neq \ell' \\
\semlab{Let} &
  \Let \; x \revto \Return \; V \; \In \; N &\reducesto& N[V/x] \\
\semlab{Handle-Ret} &
  \Handle \; (\Return \; V) \; \With \; H &\reducesto& M[V/x], \hfill\quad \text{where } \{ \Return \; x \mapsto M \} \in H \\
\semlab{Handle-Op} &
  \Handle \; \mathcal{E}[\Do \; \ell \; V] \; \With \; H
    &\reducesto& M[V/x, \lambda y . \, \Handle \; \mathcal{E}[\Return \; y] \; \With \; H/k],\qquad \\
  \multicolumn{4}{@{}r@{}}
      {\text{where } \ell \notin BL(\mathcal{E}) \text{ and } \{ \ell \; x \; k \mapsto M \} \in H} \\
\end{reductions}
\begin{syntax}
\slab{Evaluation contexts} &  \mathcal{E} &::=& [\,] \mid \Let \; x \revto \mathcal{E} \; \In \; N \mid \Handle \; \mathcal{E} \; \With \; H
\end{syntax}
\[
% Evaluation context lift
\inferrule*[Lab=\semlab{Lift}]
  { M \reducesto N }
  { \mathcal{E}[M] \reducesto \mathcal{E}[N]}
\]

\caption{Small-step Operational Semantics}
\label{fig:small-step}
\end{figure*}
We give a small-step operational semantics for \Calc. Figure
\ref{fig:small-step} displays the operational rules. The reduction
relation $\reducesto$ is defined on computation terms. The statement $M
\reducesto M'$ reads: term $M$ reduces to term $M'$ in a single
step. Most of the rules are standard. We use
%% \emph{delimited
%%   computation contexts} and
\emph{evaluation contexts} to simplify the evaluation rules, by
allowing us to focus on an active expression. The interesting rules
are the handler rules.

We write $BL(\mathcal{E})$ for the set of operation labels bound by
$\mathcal{E}$.
\begin{equations}
BL([~])                            &=& \emptyset \\
BL(\Let\;x \revto \mathcal{E}\;\In\;N)    &=& BL(\mathcal{E}) \\
BL(\Handle\;\mathcal{E}\;\With\;H) &=& BL(\mathcal{E}) \cup dom(H) \\
\end{equations}

The rule \semlab{Handle-Ret} invokes the return clause of a
handler. The rule \semlab{Handle-op} handles an operation by invoking
the appropriate operation clause. The constraint $\ell \notin
BL(\mathcal{E})$ ensures that no inner handler inside the evaluation
context is able to handle the operation: thus a handler is able to
reach past any other inner handlers that do not handle $\ell$. In our
abstract machine semantics we realise this behaviour using explicit
forwarding operations, but more efficient implementations are
perfectly feasible.


We write $R^+$ for the transitive closure of relation $R$.
%
Subject reduction and type soundness for $\Calc$ are standard.

\begin{theorem}[Subject Reduction]
If $\typc{\Delta;\Gamma}{M : A}{E}$ and $M \reducesto M'$, then
$\typc{\Delta;\Gamma}{M' : A}{E}$.
\end{theorem}

There are two ways in which a computation can terminate. It can either
successfully return a value, or it can get stuck on an unhandled
operation.
\begin{definition}
We say that computation term $N$ is normal with respect to effect $E$,
if $N$ is either of the form $\Return\;V$, or
$\mathcal{E}[\Do\;\ell\;W]$, where $\ell \in E$ and $\ell \notin
BL(\mathcal{E})$.
\end{definition}
If $N$ is normal with respect to the empty effect $\{\cdot\}$, then
$N$ has the form $\Return\;V$.

\begin{theorem}[Type Soundness]
If $\typc{}{M : A}{E}$, then there exists $\typc{}{N : A}{E}$, such that
$M \reducesto^+ N \not\reducesto$, and $N$ is normal with respect to
effect $E$.
\end{theorem}

%%
%% Conclusions and future work
%%
\chapter{Conclusions and future work}
\label{ch:conclusions}

\section{Critical evaluation}
\label{sec:criticaleval}

\section{Compiling effect handlers}
\label{sec:conclude-compiling}

\section{Effect handler enabled concurrency}
\label{sec:conclude-concurrency}

\section{Future work}
\label{sec:futurework}

%%
%% Appendices
%%
% \appendix
% \chapter{Installing the Links compiler}
% \label{ch:install}

%%%%%%%%
%% Include your chapter files here. See the sample chapter file for the basic
%% format.

%\section{Introduction}
Points:
\begin{itemize}
  \item Multicore OCaml has handlers but not an effect system
  \item Monads and monad transformers are powerful, but not really flexible, e.g. monad transformer gymnastics.
\end{itemize}

%% If you want the bibliography single-spaced (which is allowed), uncomment
%% the next line.
\nocite{*}
\singlespace
%\printbibliography[heading=bibintoc]
\bibliographystyle{abbrvnat}
\bibliography{references}

%% ... that's all, folks!
\end{document}
